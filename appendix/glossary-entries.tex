% Acronyms
\newacronym[description={\glslink{apig}{Application Program Interface}}]
    {api}{API}{Application Program Interface}

\newacronym[description={\glslink{umlg}{Unified Modeling Language}}]
    {uml}{UML}{Unified Modeling Language}

\newacronym[description={\glslink{esgg}{Environmental, Social and Governance}}]
    {esg}{ESG}{Environmental, Social and Governance}



% Glossary entries
\newglossaryentry{apig} {
    name=\glslink{api}{API},
    text=Application Program Interface,
    sort=api,
    description={in informatica con il termine \emph{Application Programming Interface API} (ing. interfaccia di programmazione di un'applicazione) si indica ogni insieme di procedure disponibili al programmatore, di solito raggruppate a formare un set di strumenti specifici per l'espletamento di un determinato compito all'interno di un certo programma. La finalità è ottenere un'astrazione, di solito tra l'hardware e il programmatore o tra software a basso e quello ad alto livello semplificando così il lavoro di programmazione}
}

\newglossaryentry{umlg} {
    name=\glslink{uml}{UML},
    text=UML,
    sort=uml,
    description={in ingegneria del software \emph{UML, Unified Modeling Language} (ing. linguaggio di modellazione unificato) è un linguaggio di modellazione e specifica basato sul paradigma object-oriented. L'\emph{UML} svolge un'importantissima funzione di ``lingua franca'' nella comunità della progettazione e programmazione a oggetti. Gran parte della letteratura di settore usa tale linguaggio per descrivere soluzioni analitiche e progettuali in modo sintetico e comprensibile a un vasto pubblico}
}

\newglossaryentry{esgg} {
    name=\glslink{esg}{ESG},
    text=ESG,
    sort=esg,
    description={\emph{Environmental, Social, Governance}, (ing. Ambientale, Sociale e di Governance) è un acronimo che indica i criteri utilizzati per valutare la sostenibilità e la responsabilità di un'azienda. L'aspetto ambientale riguarda pratiche come riduzione delle emissioni, uso delle risorse e tutela del clima; quello sociale include rapporti con dipendenti, clienti e comunità, promuovendo inclusione e condizioni di lavoro eque; infine, la governance si riferisce ai meccanismi di gestione, trasparenza, etica e correttezza nei processi decisionali.}
}