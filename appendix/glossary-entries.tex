% Acronyms
\newacronym[description={\glslink{apig}{Application Program Interface}}]
    {api}{API}{Application Program Interface}

\newacronym[description={\glslink{umlg}{Unified Modeling Language}}]
    {uml}{UML}{Unified Modeling Language}

\newacronym[description={\glslink{esgg}{Environmental, Social and Governance}}]
    {esg}{ESG}{Environmental, Social and Governance}

\newacronym[description={\glslink{aig}{Artificial Intelligence}}]
    {ai}{AI}{Artificial Intelligence}

\newacronym[description={\glslink{apmg}{Application Performance Monitoring}}]
    {apm}{APM}{Application Performance Monitoring}

\newacronym[description={\glslink{pdfg}{Portable Document Format}}]
    {pdf}{PDF}{Portable Document Format}

\newacronym[description={\glslink{csvg}{Comma-Separated Values}}]
    {csv}{CSV}{Comma-Separated Values}

\newacronym[description={\glslink{mysqlg}{My Structured Query Language}}]
    {mysql}{MySQL}{My Structured Query Language}

\newacronym[description={\glslink{httpg}{HyperText Transfer Protocol}}]
    {http}{HTTP}{HyperText Transfer Protocol}

\newacronym[description={\glslink{cpug}{Central Processing Unit}}]
    {cpu}{CPU}{Central Processing Unit}

\newacronym[description={\glslink{rumg}{Real User Monitoring}}]
    {rum}{RUM}{Real User Monitoring}

\newacronym[description={\glslink{otlpg}{OpenTelemetry Protocol}}]
    {otlp}{OTLP}{OpenTelemetry Protocol}

\newacronym[description={\glslink{eumg}{End User Monitoring}}]
    {eum}{EUM}{End User Monitoring}

\newacronym[description={\glslink{sdkg}{Software Development Kit}}]
    {sdk}{SDK}{Software Development Kit}


% Glossary entries
\newglossaryentry{apig} {
    name=\glslink{api}{API},
    text=Application Program Interface,
    sort=api,
    description={in informatica con il termine \emph{Application Programming Interface API} (ing. interfaccia di programmazione di un'applicazione) si indica ogni insieme di procedure disponibili al programmatore, di solito raggruppate a formare un set di strumenti specifici per l'espletamento di un determinato compito all'interno di un certo programma. La finalità è ottenere un'astrazione, di solito tra l'hardware e il programmatore o tra software a basso e quello ad alto livello semplificando così il lavoro di programmazione}
}

\newglossaryentry{umlg} {
    name=\glslink{uml}{UML},
    text=UML,
    sort=uml,
    description={in ingegneria del software \emph{UML, Unified Modeling Language} (ing. linguaggio di modellazione unificato) è un linguaggio di modellazione e specifica basato sul paradigma object-oriented. L'\emph{UML} svolge un'importantissima funzione di ``lingua franca'' nella comunità della progettazione e programmazione a oggetti. Gran parte della letteratura di settore usa tale linguaggio per descrivere soluzioni analitiche e progettuali in modo sintetico e comprensibile a un vasto pubblico}
}

\newglossaryentry{esgg} {
    name=\glslink{esg}{ESG},
    text=ESG,
    sort=esg,
    description={\emph{Environmental, Social, Governance}, (ing. Ambientale, Sociale e di Governance) è un acronimo che indica i criteri utilizzati per valutare la sostenibilità e la responsabilità di un'azienda. L'aspetto ambientale riguarda pratiche come riduzione delle emissioni, uso delle risorse e tutela del clima; quello sociale include rapporti con dipendenti, clienti e comunità, promuovendo inclusione e condizioni di lavoro eque; infine, la governance si riferisce ai meccanismi di gestione, trasparenza, etica e correttezza nei processi decisionali}
}

\newglossaryentry{aig} {
    name=\glslink{ai}{AI},
    text=AI,
    sort=ai,
    description={\emph{Artificial Intelligence} (ing. Intelligenza Artificiale) è un ramo dell'informatica che si occupa della creazione di sistemi in grado di svolgere compiti che normalmente richiederebbero l'intelligenza umana, come il riconoscimento vocale, la visione artificiale, l'apprendimento automatico e la risoluzione di problemi complessi. L'obiettivo principale dell'AI è sviluppare algoritmi e modelli che permettano alle macchine di apprendere dai dati, adattarsi a nuove situazioni e prendere decisioni autonome, migliorando così l'efficienza e l'efficacia in vari settori, tra cui la medicina, la finanza, l'industria e i servizi}
}

\newglossaryentry{apmg} {
    name=\glslink{apm}{APM},
    text=APM,
    sort=apm,
    description={\emph{Application Performance Monitoring} (ing. Monitoraggio delle Prestazioni delle Applicazioni) è un insieme di pratiche e strumenti utilizzati per monitorare, misurare e gestire le prestazioni e la disponibilità delle applicazioni software. L'obiettivo principale dell'APM è garantire che le applicazioni funzionino in modo ottimale, offrendo un'esperienza utente fluida e senza interruzioni. Ciò include il monitoraggio di vari parametri come tempi di risposta, tassi di errore, utilizzo delle risorse e throughput, nonché l'identificazione e la risoluzione di problemi che potrebbero influire sulle prestazioni dell'applicazione}
}

\newglossaryentry{pdfg} {
    name=\glslink{pdf}{PDF},
    text=PDF,
    sort=pdf,
    description={\emph{Portable Document Format} (ing. Formato di Documento Portatile) è un formato di file sviluppato da Adobe Systems per rappresentare documenti in modo indipendente dall'hardware, dal software e dal sistema operativo utilizzati per crearli o visualizzarli. I file PDF possono contenere testo, immagini, grafica vettoriale e persino elementi interattivi come moduli compilabili e collegamenti ipertestuali. Il formato PDF è ampiamente utilizzato per la condivisione di documenti, in quanto preserva il layout e la formattazione originale, garantendo che il documento venga visualizzato correttamente su qualsiasi dispositivo o piattaforma}
}

\newglossaryentry{csvg} {
    name=\glslink{csv}{CSV},
    text=CSV,
    sort=csv,
    description={\emph{Comma-Separated Values} (ing. Valori Separati da Virgola) è un formato di file di testo utilizzato per rappresentare dati tabulari, in cui ogni riga del file corrisponde a un record e i valori all'interno di ogni record sono separati da virgole. Il formato CSV è ampiamente utilizzato per l'importazione e l'esportazione di dati tra diversi programmi, come fogli di calcolo, database e applicazioni di analisi dei dati, grazie alla sua semplicità e compatibilità con molti software}
}

\newglossaryentry{mysqlg} {
    name=\glslink{mysql}{MySQL},
    text=MySQL,
    sort=mysql,
    description={\emph{My Structured Query Language} è un sistema di gestione di database relazionali (RDBMS) open source basato sul linguaggio SQL (Structured Query Language). MySQL è ampiamente utilizzato per la gestione e l'organizzazione di grandi quantità di dati in applicazioni web, software aziendali e sistemi di gestione dei contenuti. Offre funzionalità avanzate come transazioni, replica, partizionamento e supporto per vari motori di archiviazione, rendendolo una scelta popolare per sviluppatori e amministratori di database in tutto il mondo}
}

\newglossaryentry{httpg} {
    name=\glslink{http}{HTTP},
    text=HTTP,
    sort=http,
    description={\emph{HyperText Transfer Protocol} (ing. Protocollo di Trasferimento Ipertestuale) è un protocollo di comunicazione utilizzato per la trasmissione di dati su Internet, in particolare per il trasferimento di pagine web tra server e client (come i browser web). HTTP definisce le regole e le convenzioni per la richiesta e la risposta di risorse, come documenti HTML, immagini, video e altri contenuti multimediali. Il protocollo è basato su un modello client-server, in cui il client invia una richiesta al server, che elabora la richiesta e restituisce una risposta contenente i dati richiesti}
}

\newglossaryentry{cpug} {
    name=\glslink{cpu}{CPU},
    text=CPU,
    sort=cpu,
    description={\emph{Central Processing Unit} (ing. Unità Centrale di Elaborazione), comunemente nota come processore, è il componente principale di un computer responsabile dell'esecuzione delle istruzioni dei programmi e del controllo delle operazioni del sistema. La CPU interpreta e processa i dati, eseguendo operazioni aritmetiche, logiche e di controllo, fungendo da cervello del computer. Le prestazioni della CPU sono influenzate da vari fattori, tra cui la velocità di clock, il numero di core e l'architettura del processore}
}

\newglossaryentry{rumg} {
    name=\glslink{rum}{RUM},
    text=RUM,
    sort=rum,
    description={\emph{Real User Monitoring} (ing. Monitoraggio degli Utenti Reali) è una tecnica di monitoraggio delle prestazioni delle applicazioni web che si concentra sull'analisi del comportamento e dell'esperienza degli utenti reali mentre interagiscono con un sito web o un'applicazione. A differenza delle soluzioni di monitoraggio sintetico, che simulano il traffico utente attraverso script predefiniti, il RUM raccoglie dati direttamente dai browser degli utenti, fornendo informazioni dettagliate sulle prestazioni percepite, i tempi di caricamento delle pagine, gli errori riscontrati e altri aspetti critici dell'esperienza utente}
}

\newglossaryentry{otlpg} {
    name=\glslink{otlp}{OTLP},
    text=OTLP,
    sort=otlp,
    description={\emph{OpenTelemetry Protocol} (ing. Protocollo OpenTelemetry) è un protocollo di comunicazione standardizzato utilizzato per la trasmissione di dati di telemetria, come tracce, metriche e log, tra componenti di sistemi di osservabilità basati su OpenTelemetry. Il protocollo supporta vari formati di trasporto, facilitando l'integrazione con una vasta gamma di tecnologie e infrastrutture}
}

\newglossaryentry{eumg} {
    name=\glslink{eum}{EUM},
    text=EUM,
    sort=eum,
    description={\emph{End User Monitoring} (ing. Monitoraggio dell'Utente Finale) è una tecnica di monitoraggio delle prestazioni delle applicazioni che si concentra sull'analisi dell'esperienza degli utenti finali mentre interagiscono con un'applicazione o un sistema. L'EUM raccoglie dati in tempo reale sulle prestazioni percepite dagli utenti, come i tempi di risposta, la disponibilità del servizio e gli errori riscontrati, fornendo informazioni preziose per migliorare l'usabilità e l'efficienza dell'applicazione}
}

\newglossaryentry{sdkg} {
    name=\glslink{sdk}{SDK},
    text=SDK,
    sort=sdk,
    description={\emph{Software Development Kit} (ing. Kit di Sviluppo Software) è un insieme di strumenti, librerie, documentazione e esempi di codice forniti agli sviluppatori per facilitare la creazione di applicazioni software specifiche per una piattaforma, un framework o un sistema operativo}
}