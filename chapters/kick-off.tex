\chapter{Descrizione dello stage}
\label{cap:descrizione-stage}

\intro{Il capitolo approfondisce il progetto di stage, descrivendo gli obiettivi e le attività svolte, la metodologia di lavoro adottata e un'analisi preventiva dei principali rischi e relative strategie di mitigazione.}\\

\section{Introduzione al progetto}
Lo stage è stato svolto presso l'azienda Kirey Group S.r.l., realtà consolidata nell'ambito della fornitura di prodotti e servizi informatici, con clienti internazionali e una forte specializzazione nei settori \emph{Banking}, \emph{Insurance}, \emph{Oil\&Gas} e Pubblica Amministrazione. \\ 
L'attività si è inserita nel contesto del team \gls{apmg}\glsfirstoccur e ha avuto come obiettivo principale la realizzazione e il collaudo di una piattaforma per il monitoraggio delle performance di una \emph{web application}. \\
Il progetto è stato sviluppato interamente in ambiente \emph{Linux} mediante \emph{Windows Subsystem for Linux}, utilizzando la \emph{suite} \emph{Elastic Stack} e i suoi principali componenti: \emph{Elasticsearch} per la gestione dei dati, \emph{Kibana} per la visualizzazione, \emph{Logstash} per l'ingestione e la trasformazione dei log, \emph{Beats} e \emph{Fleet Server} per la raccolta distribuita delle metriche e \emph{APM Server/Agent} per il tracciamento delle \emph{performance} applicative. \\ 
Sono stati realizzati sviluppi in \emph{Python} e \emph{Java} per l'estensione delle funzionalità e l'integrazione di algoritmi di \gls{aig}\glsfirstoccur e \emph{Machine Learning}, con l'obiettivo di rilevare automaticamente anomalie e problematiche di prestazione. \\   
La finalità complessiva è quella di fornire un sistema scalabile, proattivo e ben documentato, capace di garantire prestazioni ottimali e un monitoraggio continuo della \emph{web application}.  


\section{Pianificazione}
Tutte le attività sono state condotte in affiancamento ad un tutor aziendale che ha curato sia la parte di formazione che di indirizzamento delle attività.
A tal fine sono stati svolti dei momenti di confronto settimanali per la valutazione dello stato di avanzamento delle attività e momenti quotidiani di confronto sulle problematiche riscontrate. \\
Le attività proposte sono state collocate all'interno di un progetto più ampio portato avanti in Kirey da un team di persone eterogeneo. \\
Al termine dello stage sono stati presentati i risultati ottenuti a tutto il team.
L'infrastruttura tecnologica e le piattaforme su cui girerà l'applicazione sono state messe a disposizione da Kirey.


\section{Analisi preventiva dei rischi}

Durante la fase di analisi iniziale sono stati individuati alcuni possibili rischi a cui si potrà andare incontro.
Si è quindi proceduto a elaborare delle possibili soluzioni per far fronte a tali rischi.\\

\begin{risk}{Inesperienza nella suite Elastic}
    \vspace{0.5em}
    \riskdescription{La limitata esperienza iniziale nell'utilizzo della \emph{Elastic Stack} (\emph{Elasticsearch}, \emph{Kibana}, \emph{Logstash}, \emph{APM}) potrebbe comportare difficoltà nella configurazione e nell'integrazione delle componenti, rallentando lo sviluppo del progetto}
    \vspace{0.5em}
    \riskimpact{Alto}
    \vspace{0.5em}
    \riskprobability{Alta}
    \vspace{0.5em}
    \risksolution{Organizzazione di momenti di confronto con il tutor aziendale e studio personale della documentazione, al fine di acquisire le competenze necessarie}
    \label{risk:elastic-inexperience}
\end{risk}

\vspace{1em}

\begin{risk}{Integrazione tra componenti Elastic}
    \vspace{0.5em}
    \riskdescription{La comunicazione tra i diversi moduli della \emph{suite} \emph{Elastic} (\emph{Elasticsearch}, \emph{Kibana}, \emph{Logstash}, \emph{APM}) potrebbe presentare problemi di configurazione, causando ritardi o malfunzionamenti nell'acquisizione dei dati}
    \vspace{0.5em}
    \riskimpact{Medio}
    \vspace{0.5em}
    \riskprobability{Media}
    \vspace{0.5em}
    \risksolution{Esecuzione di test di connettività e validazione progressiva delle \emph{pipeline}, con il supporto del \emph{tutor} aziendale per la risoluzione dei problemi}
    \label{risk:elastic-integration}
\end{risk}

\vspace{1em}

\begin{risk}{Qualità e coerenza dei dati raccolti}
    \vspace{0.5em}
    \riskdescription{I dati acquisiti dagli agenti potrebbero risultare incompleti, duplicati o non coerenti, compromettendo le analisi e le \emph{dashboard}}
    \vspace{0.5em}
    \riskimpact{Alto}
    \vspace{0.5em}
    \riskprobability{Media}
    \vspace{0.5em}
    \risksolution{Definizione di regole di filtraggio e validazione all'interno delle \emph{pipeline} \emph{Logstash} ed esecuzione di test di integrità sugli indici \emph{Elasticsearch}}
    \label{risk:data-quality}
\end{risk}

\vspace{1em}

\begin{risk}{Scalabilità e carico del sistema}
    \vspace{0.5em}
    \riskdescription{L'aumento del volume dei dati e delle richieste potrebbe impattare sulle prestazioni della piattaforma, riducendo l'efficienza del monitoraggio}
    \vspace{0.5em}
    \riskimpact{Medio}
    \vspace{0.5em}
    \riskprobability{Media}
    \vspace{0.5em}
    \risksolution{Implementazione di strategie di \emph{scaling} orizzontale e utilizzo di metriche di \emph{Kibana} per monitorare l'impatto del carico in tempo reale}
    \label{risk:scalability}
\end{risk}

\vspace{1em}

\begin{risk}{Implementazione di algoritmi di Machine Learning}
    \vspace{0.5em}
    \riskdescription{L'integrazione di modelli di \emph{Machine Learning} per il rilevamento delle anomalie potrebbe richiedere competenze specifiche e tempi di sviluppo più lunghi del previsto}
    \vspace{0.5em}
    \riskimpact{Alto}
    \vspace{0.5em}
    \riskprobability{Media}
    \vspace{0.5em}
    \risksolution{Formazione preliminare su tecniche di \emph{Machine Learning} e utilizzo di librerie e \emph{framework} consolidati per accelerare lo sviluppo}
    \label{risk:ml-implementation}
\end{risk}

\vspace{1em}

\begin{risk}{Problemi di configurazione delle pipeline Logstash}
    \vspace{0.5em}
    \riskdescription{Errori di configurazione nelle \emph{pipeline} di \emph{Logstash} potrebbero causare la perdita, la duplicazione o la trasformazione errata dei dati raccolti, compromettendo l'affidabilità delle analisi}
    \vspace{0.5em}
    \riskimpact{Alto}
    \vspace{0.5em}
    \riskprobability{Media}
    \vspace{0.5em}
    \risksolution{Adozione di un approccio con \emph{test} di validazione a ogni modifica delle \emph{pipeline} e utilizzo di ambienti di prova per verificare la correttezza del flusso dei dati prima della messa in produzione}
    \label{risk:logstash-pipeline}
\end{risk}

\vspace{1em}

\begin{risk}{Accesso limitato a funzionalità premium di Elastic}
    \vspace{0.5em}
    \riskdescription{Durante le prime settimane di lavoro in locale, la mancata possibilità di operare su \emph{Elastic Cloud} e di accedere a funzionalità premium come il \emph{Machine Learning} potrebbe limitare l'analisi dei dati e rallentare la validazione di alcune funzionalità previste dal progetto}
    \vspace{0.5em}
    \riskimpact{Medio}
    \vspace{0.5em}
    \riskprobability{Bassa}
    \vspace{0.5em}
    \risksolution{Esecuzione preventiva delle attività in locale con dati di test, studio della documentazione sulle funzionalità premium e pianificazione di un passaggio successivo a \emph{Elastic Cloud} non appena disponibile, con supporto del tutor aziendale}
    \label{risk:elastic-cloud-limitations}
\end{risk}


\section{Requisiti e obiettivi}
Gli obiettivi del progetto sono stati classificati in base alla loro priorità secondo le seguenti notazioni:

\begin{itemize}
    \item \textbf{Ob} (Requisiti Obbligatori) - requisiti essenziali e imprescindibili per il successo del progetto, vincolanti in quanto obiettivo primario richiesto dal committente;
    \item \textbf{D} (Requisiti Desiderabili) - requisiti importanti ma non critici, la cui assenza non compromette il progetto, non vincolanti o strettamente necessari, ma dal riconoscibile valore aggiunto;
    \item \textbf{Op} (Requisiti Opzionali) - requisiti desiderabili ma non essenziali, rappresentanti valore aggiunto non strettamente competitivo.
\end{itemize}

\vspace{1em}

Durante il periodo di stage si prevede il raggiungimento dei seguenti obiettivi:

\begin{itemize}
    \item \textbf{Preparazione dell'ambiente per l'implementazione della soluzione:}  
        \begin{itemize}
            \item \textbf{Ob1.1:} Individuazione delle componenti ed eventuali librerie da utilizzare;
            \item \textbf{Ob1.2:} Installazione e configurazione delle componenti;
            \item \textbf{Ob1.3:} Verifica del corretto funzionamento dell’ambiente e test di connettività tra componenti.
        \end{itemize}
    \item \textbf{Implementazione estrazione dati dalla web application:}  
        \begin{itemize}
            \item \textbf{Ob2.1:} Configurazione \emph{agent} (\emph{Beats/APM}) per la raccolta dati della navigazione;
            \item \textbf{Ob2.2:} Implementazione \emph{pipeline} di \emph{log} tramite \emph{Logstash} per filtraggio e inoltro dati in \emph{Elasticsearch};
            \item \textbf{Ob2.3:} Verifica della corretta acquisizione dei dati e loro indicizzazione in \emph{Elasticsearch}.
        \end{itemize}
    \item \textbf{Implementazione elaborazione dati e rappresentazione grafica dei dati:}  
        \begin{itemize}
            \item \textbf{Ob3.1:} Sviluppo \emph{script} \emph{Python/Java} per il monitoraggio sintetico (\emph{Selenium}) e generazione di traffico \emph{log};
            \item \textbf{Ob3.2:} Analisi e aggregazione dati in \emph{Elasticsearch}, con \emph{query} e visualizzazioni preliminari;
            \item \textbf{Ob3.3:} Creazione dashboard avanzate su \emph{Kibana} con metriche di \emph{performance}, accesso e flussi utente;
            \item \textbf{Op3.4:} Configurazione regole di \emph{alerting} e notifiche per anomalie rilevate in tempo reale.
        \end{itemize}
    \item \textbf{Documentazione dettagliata:}  
        \begin{itemize}
            \item \textbf{Ob4.1:} Descrizione delle tecnologie e prodotti utilizzati;
            \item \textbf{Ob4.2:} Descrizione dei flussi logici del progetto e delle funzionalità dell'applicazione;
            \item \textbf{D4.3:} Pro/contro di ogni componente e criticità nell'applicazione.
        \end{itemize}
\end{itemize} 