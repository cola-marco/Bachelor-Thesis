\chapter{Introduzione}
\label{cap:introduzione}

\section{L'azienda}

\begin{figure}[!h] 
    \centering 
    \includegraphics[width=0.9\columnwidth]{Kirey_logo.jpg} 
    \caption{Figura 1 - Logo di Kirey S.r.l.}
\end{figure}

Kirey Group è uno dei \emph{system integrator} europei più dinamici e in crescita, specializzato nell'accompagnare le imprese nei percorsi di trasformazione digitale
 e di adozione di modelli \emph{data-driven}. Con sede principale in Italia e una presenza consolidata in diversi Paesi europei ed extraeuropei, 
 il Gruppo conta oltre 1000 professionisti e opera in dieci Paesi.

La missione di Kirey è rendere l'innovazione accessibile, trasformando il potenziale tecnologico in valore economico e in nuovi modelli di business. 
L'azienda si distingue per un approccio che unisce affidabilità tecnica, innovazione guidata dai dati, competenza centrata sul lavoro delle persone e sinergia cross-funzionale, 
elementi che costituiscono i valori fondanti del marchio.

Il manifesto del gruppo sintetizza questa filosofia nel concetto “\emph{Data Made Human}”, ovvero la volontà di tradurre la complessità dei dati in soluzioni comprensibili, 
intuitive e ad alto impatto, mettendo sempre la persona al centro della tecnologia.

La storia del gruppo affonda le radici negli anni Settanta e, attraverso fusioni, acquisizioni e nuove fondazioni, ha portato alla nascita di Kirey Group nel 2016. 
Negli anni successivi l'azienda ha accelerato la propria espansione internazionale integrando nuove realtà, consolidando così competenze e capacità operative in diversi settori e mercati.

Il portafoglio di servizi è ampio e integrato, con \emph{Data \& AI} come filo conduttore e aree principali che comprendono:

\begin{itemize}
    \item \emph{Cloud \& Infrastructure}, con soluzioni ibride e \emph{on-premise}, sicurezza in ambienti \emph{cloud}, migrazione e monitoraggio;
    \item \emph{Software Development}, che spazia dallo sviluppo agile e mobile alla \emph{system integration}, con particolare attenzione alla qualità e all'automazione dei \emph{test};
    \item \emph{Cybersecurity}, con servizi di consulenza, \emph{audit}, architetture sicure, \emph{managed services} e sistemi antifrode;
    \item \emph{Data \& AI}, che include \emph{data integration}, \emph{data governance}, \emph{analytics}, \emph{machine learning}, \emph{synthetic data}, \emph{forecasting} e soluzioni \gls{esg}.
\end{itemize}

Kirey Group pone grande attenzione alla sostenibilità, alla trasparenza e all'integrità, adottando pratiche responsabili nei confronti di clienti, partner, dipendenti e \emph{stakeholder}. 
L'azienda è inoltre attivamente impegnata in progetti sociali, promuove la diversità e l'inclusione, e investe nello sviluppo delle competenze tecnologiche e professionali delle proprie persone.

Oggi il gruppo conta oltre 1370 casi di business realizzati, 10 \emph{Innovation Center} attivi, un fatturato di circa 126 milioni di euro e più di 1000 collaboratori distribuiti in 10 paesi.


\section{L'idea}

L'idea alla base dello stage consiste nello sviluppo di una piattaforma per il monitoraggio intelligente del traffico utente di un e-commerce. L'obiettivo principale è quello di sfruttare algoritmi di Intelligenza Artificiale e di Machine Learning per individuare e segnalare automaticamente eventuali anomalie nei dati raccolti.

La piattaforma sarà in grado di analizzare i flussi in tempo reale, rilevando accessi sospetti, rallentamenti e potenziali minacce, così da consentire interventi tempestivi e garantire sia la sicurezza sia le prestazioni ottimali del sistema.

Il progetto sarà articolato in diverse fasi: una prima fase di formazione, utile a familiarizzare con le tecnologie e i prodotti utilizzati; una fase di analisi e progettazione, in cui saranno definite le specifiche funzionali e la soluzione tecnica; una fase di realizzazione e test della piattaforma; e infine la stesura della documentazione tecnica e funzionale. 

Per lo sviluppo saranno impiegati linguaggi come Python e Java, sistemi Linux e i prodotti della suite Elastic Stack, strumenti particolarmente adatti per l'elaborazione e il monitoraggio di grandi volumi di dati in tempo reale.


\section{Organizzazione del testo}

\begin{description}
    \item[{\hyperref[cap:processi-metodologie]{Il secondo capitolo}}] descrive ...
    
    \item[{\hyperref[cap:descrizione-stage]{Il terzo capitolo}}] approfondisce ...
    
    \item[{\hyperref[cap:analisi-requisiti]{Il quarto capitolo}}] approfondisce ...
    
    \item[{\hyperref[cap:progettazione-codifica]{Il quinto capitolo}}] approfondisce ...
    
    \item[{\hyperref[cap:verifica-validazione]{Il sesto capitolo}}] approfondisce ...
    
    \item[{\hyperref[cap:conclusioni]{Nel settimo capitolo}}] descrive ...
\end{description}

Riguardo la stesura del testo, relativamente al documento sono state adottate le seguenti convenzioni tipografiche:
\begin{itemize}
	\item gli acronimi, le abbreviazioni e i termini ambigui o di uso non comune menzionati vengono definiti nel glossario, situato alla fine del presente documento;
	\item per la prima occorrenza dei termini riportati nel glossario viene utilizzata la seguente nomenclatura: \emph{parola}\glsfirstoccur;
	\item i termini in lingua straniera o facenti parti del gergo tecnico sono evidenziati con il carattere \emph{corsivo}.
\end{itemize}
