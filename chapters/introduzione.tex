\chapter{Introduzione}
\label{cap:introduzione}

\section{L'azienda}

Kirey Group è uno dei \emph{system integrator} europei più dinamici e in crescita, specializzato nell'accompagnare le imprese nei percorsi di trasformazione digitale e di adozione di modelli \emph{data-driven}. \\
Con sede principale in Italia e una presenza consolidata in diversi paesi europei ed extraeuropei, il gruppo conta oltre 1500 dipendenti ed opera in dieci paesi. \\
La missione di Kirey è rendere l'innovazione accessibile, trasformando il potenziale tecnologico in valore economico e in nuovi modelli di \emph{business}. \\
L'azienda si distingue per un approccio che unisce affidabilità tecnica, innovazione, competenza centrata sul lavoro delle persone e sinergia cross-funzionale, elementi che costituiscono i valori fondanti del marchio. \\
Il manifesto del gruppo sintetizza questa filosofia nel concetto “\emph{Data Made Human}”, ovvero la volontà di tradurre la complessità dei dati in soluzioni comprensibili, intuitive e ad alto impatto, mettendo sempre la persona al centro della tecnologia.

\begin{figure}[!h] 
    \centering 
    \includegraphics[width=0.9\columnwidth]{Kirey_logo.jpg} 
    \caption{Logo di Kirey S.r.l.}
\end{figure}


\subsection{Storia e servizi}

La storia del gruppo affonda le radici negli anni settanta e, attraverso fusioni, acquisizioni e nuove fondazioni, ha portato alla nascita di Kirey Group nel 2016. \\ 
Negli anni successivi l'azienda ha accelerato la propria espansione internazionale integrando nuove realtà, consolidando così competenze e capacità operative in diversi settori e mercati. \\
Il portafoglio di servizi è ampio e integrato, con \emph{Data \& AI} come filo conduttore e aree principali che comprendono:

\begin{itemize}
    \item \emph{Cloud \& Infrastructure}, con soluzioni ibride e \emph{on-premise}, sicurezza in ambienti \emph{cloud}, migrazione e monitoraggio;
    \item \emph{Software Development}, che spazia dallo sviluppo \emph{agile} e \emph{mobile} alla \emph{system integration}, con particolare attenzione alla qualità e all'automazione dei \emph{test};
    \item \emph{Cybersecurity}, con servizi di consulenza, \emph{audit}, architetture sicure, \emph{managed services} e sistemi antifrode;
    \item \emph{Data \& AI}, che include \emph{data integration}, \emph{data governance}, \emph{analytics}, \emph{machine learning}, \emph{synthetic data}, \emph{forecasting} e soluzioni \gls{esgg}\glsfirstoccur.
\end{itemize}
Kirey Group pone grande attenzione alla sostenibilità, alla trasparenza e all'integrità, adottando pratiche responsabili nei confronti di clienti, partner, dipendenti e \emph{stakeholder}. \\
L'azienda è inoltre attivamente impegnata in progetti sociali, promuove la diversità e l'inclusione, e investe nello sviluppo delle competenze tecnologiche e professionali dei propri collaboratori. \\
Oggi il gruppo conta oltre 1370 casi di \emph{business} realizzati, 10 \emph{Innovation Center} attivi, un fatturato di circa 126 milioni di euro e più di 1500 collaboratori distribuiti in 10 paesi.


\section{L'idea}

L'idea alla base dello \emph{stage} consiste nello sviluppo di una piattaforma per il monitoraggio intelligente del traffico utente di una \emph{web application}. L'obiettivo principale è quello di sfruttare algoritmi di intelligenza artificiale e di \emph{Machine Learning} per individuare e segnalare automaticamente eventuali anomalie nei dati raccolti. \\
La piattaforma sarà in grado di analizzare i flussi in tempo reale, rilevando accessi sospetti, rallentamenti e potenziali minacce, così da consentire interventi tempestivi e garantire sia la sicurezza sia le prestazioni ottimali del sistema. \\
Il progetto è stato organizzato in quattro fasi principali:

\begin{itemize}
    \item Una prima fase di formazione e preparazione dell'ambiente di lavoro, utile a familiarizzare con le tecnologie e i prodotti utilizzati, che comprende l'individuazione, l'installazione e la configurazione delle componenti necessarie, con successiva verifica della connettività tra i moduli;
    \item Una fase di analisi e progettazione, in cui saranno definite le specifiche funzionali e la soluzione tecnica tramite la configurazione degli agenti di raccolta, la realizzazione di \emph{pipeline} di \emph{log} e la loro indicizzazione in \emph{Elasticsearch};
    \item Una fase di realizzazione e \emph{test} della piattaforma, con sviluppo di \emph{script} per il monitoraggio sintetico, analisi su \emph{Elasticsearch}, creazione di \emph{dashboard} in \emph{Kibana} e configurazione di regole di \emph{alerting};
    \item Infine la stesura della documentazione tecnica e funzionale, contenente la descrizione delle tecnologie adottate, dei flussi logici implementati e delle criticità riscontrate.
\end{itemize}
Per lo sviluppo saranno impiegati linguaggi come \emph{Python} e \emph{Java}, sistemi \emph{Linux} e i prodotti della suite \emph{Elastic Stack}, strumenti particolarmente adatti per l'elaborazione e il monitoraggio di grandi volumi di dati in tempo reale.


\section{Organizzazione del testo}

\begin{description}
    \item[{\hyperref[cap:descrizione-stage]{Il secondo capitolo}}] approfondisce il progetto di \emph{stage}, descrivendo gli obiettivi e le attività svolte, la metodologia di lavoro adottata e un'analisi preventiva dei principali rischi e relative strategie di mitigazione;

    \item[{\hyperref[cap:analisi-requisiti]{Il terzo capitolo}}] è dedicato all'analisi dei requisiti della piattaforma e alle motivazioni che ne hanno guidato le scelte progettuali, fornendo al contempo una mappatura completa e strutturata dei requisiti individuati per il sistema;

    \item[{\hyperref[cap:tecnologie-principi-teorici]{Il quarto capitolo}}] approfondisce le tecnologie e i principi teorici alla base della soluzione proposta, analizzando i principali strumenti e approcci per il monitoraggio delle prestazioni applicative;

    \item[{\hyperref[cap:product-development]{Il quinto capitolo}}] approfondisce lo sviluppo del prodotto, descrivendo l'architettura complessiva del sistema, le fasi di implementazione e i risultati ottenuti;

    \item[{\hyperref[cap:conclusioni]{Nel sesto capitolo}}] viene descritta l'esperienza complessiva dello \emph{stage}, con una riflessione sui risultati raggiunti e le competenze acquisite, oltre a suggerimenti per possibili sviluppi futuri del progetto.
\end{description}

\subsection{Convenzioni tipografiche}

Riguardo la stesura del testo, relativamente al documento sono state adottate le seguenti convenzioni tipografiche:
\begin{itemize}
	\item gli acronimi, le abbreviazioni e i termini ambigui o di uso non comune menzionati vengono definiti nel glossario, situato alla fine del presente documento;
	\item per la prima occorrenza dei termini riportati nel glossario viene utilizzata la seguente nomenclatura: \emph{parola}\glsfirstoccur;
	\item i termini in lingua straniera o facenti parti del gergo tecnico sono evidenziati con il carattere \emph{corsivo};
	\item i comandi, le query, i percorsi di file e il codice inline sono rappresentati tramite il carattere \texttt{monospaziato}.
\end{itemize}
