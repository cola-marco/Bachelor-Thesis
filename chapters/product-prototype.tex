\chapter{Tecnologie e principi teorici}
\label{cap:tecnologie-principi-teorici}

\intro{Il capitolo presenta il quadro teorico e tecnologico di riferimento del progetto, descrivendo i principali approcci e strumenti per il monitoraggio delle prestazioni applicative. Vengono illustrate le tecnologie analizzate e le motivazioni che hanno guidato la scelta della soluzione, in relazione ai requisiti e agli obiettivi del sistema di osservabilità sviluppato.}\\

\section{APM e Observability}
\label{sec:apm-observability}
Definizione, Obiettivi e componenti tipici di un sistema APM.


\section{Approcci e tecnologie per il monitoraggio}
\label{sec:approcci-tecnologie-monitoraggio}

\subsection{Approcci principali}


\subsection{Tecnologie e piattaforme note}


\section{Tecnologie e strumenti}
\label{sec:tecnologie-strumenti}
Di seguito viene data una panoramica delle tecnologie e degli strumenti utilizzati.

\subsection{Elastic Stack}

\subsection*{Elasticsearch}
Elasticsearch è un motore di ricerca e analisi distribuito basato su \emph{Apache Lucene}, progettato per gestire grandi volumi di dati in tempo reale. Fornisce funzionalità avanzate di ricerca \emph{full-text}, analisi dei dati e aggregazioni, rendendolo ideale per applicazioni di monitoraggio, analisi dei \emph{log} e \emph{business intelligence}. Elasticsearch supporta un'architettura scalabile e flessibile, consentendo di distribuire i dati su più nodi e cluster per garantire alta disponibilità e prestazioni elevate.

\begin{figure}[H] 
    \centering 
    \includegraphics[width=4cm]{/logos/elasticsearch.png} 
    \caption{Figura 4.1 - Elasticsearch}
\end{figure}


\subsection*{Kibana}
Kibana è uno strumento di visualizzazione e analisi dei dati open source, progettato per lavorare in stretta integrazione con Elasticsearch. Fornisce un'interfaccia utente intuitiva che consente agli utenti di creare \emph{dashboard} interattive, visualizzare grafici, mappe e tabelle, e analizzare i dati in tempo reale. Kibana supporta una vasta gamma di visualizzazioni personalizzabili e offre funzionalità avanzate come il filtraggio dei dati, la ricerca \emph{full-text} e l'esplorazione delle relazioni tra i dati, rendendolo uno strumento potente per l'analisi dei \emph{log}, il monitoraggio delle prestazioni e la \emph{business intelligence}.

\begin{figure}[H] 
    \centering 
    \includegraphics[width=4cm]{/logos/kibana.png} 
    \caption{Figura 4.2 - Kibana}
\end{figure}


\subsection*{Beats}
Beats è una piattaforma leggera di raccolta dati open source sviluppata da Elastic, progettata per inviare dati da diverse fonti a Elasticsearch o Logstash. I \emph{Beats} sono agenti specializzati che raccolgono specifici tipi di dati, come \emph{log} di sistema, metriche di rete, dati di sicurezza e altro ancora. Grazie alla loro architettura modulare e alla facilità di configurazione, i \emph{Beats} consentono agli utenti di monitorare e analizzare rapidamente i dati provenienti da ambienti distribuiti e complessi.


\subsection*{Elastic Agent e Fleet}
Elastic Agent è un agente unificato sviluppato da Elastic che consente di raccogliere, monitorare e proteggere i dati da diverse fonti in modo semplice ed efficiente. Integrando funzionalità di raccolta dati, sicurezza e monitoraggio, Elastic Agent semplifica la gestione degli agenti e riduce la complessità operativa. Fleet è una funzionalità di gestione centralizzata all'interno di Kibana che consente di distribuire, configurare e monitorare gli Elastic Agent in modo scalabile e automatizzato. Attraverso Fleet, gli utenti possono gestire facilmente le policy di raccolta dati, aggiornare gli agenti e monitorare lo stato della loro infrastruttura da un'unica interfaccia.


\subsection*{OpenTelemetry}

\begin{figure}[H] 
    \centering 
    \includegraphics[width=4cm]{/logos/OpenTelemetry.png} 
    \caption{Figura 4.3 - OpenTelemetry}
\end{figure}

\subsection*{Elastic APM e APM Server}
\begin{figure}[H] 
    \centering 
    \includegraphics[width=4cm]{/logos/ElasticAPM.png} 
    \caption{Figura 4.4 - Elastic APM}
\end{figure}

\subsection*{RUM Agent e Synthetic Monitoring}

\subsection*{MCP}


\subsection{Linguaggi di programmazione}
\subsection*{Python}
\begin{figure}[H] 
    \centering 
    \includegraphics[width=4cm]{/logos/Python_logonormal.png} 
    \caption{Figura 4.5 - Python}
\end{figure}


\subsection*{JavaScript}
\begin{figure}[H] 
    \centering 
    \includegraphics[width=4cm]{/logos/javascript.png} 
    \caption{Figura 4.6 - JavaScript}
\end{figure}


\subsection*{Java}
\begin{figure}[H] 
    \centering 
    \includegraphics[width=4cm]{/logos/java.png} 
    \caption{Figura 4.7 - Java}
\end{figure}


\subsection{Framework e librerie}
\subsection*{Node.js}
\begin{figure}[H] 
    \centering 
    \includegraphics[width=4cm]{/logos/nodeJS.png} 
    \caption{Figura 4.8 - Node.js}
\end{figure}


\subsection*{Spring Boot}
\begin{figure}[H] 
    \centering 
    \includegraphics[width=4cm]{/logos/spring.png} 
    \caption{Figura 4.9 - Spring}
\end{figure}


\subsection*{Selenium}
\begin{figure}[H] 
    \centering 
    \includegraphics[width=4cm]{/logos/selenium.png} 
    \caption{Figura 4.10 - Selenium}
\end{figure}


\subsection{Strumenti di sviluppo}
\subsection*{VSCode}
\begin{figure}[H] 
    \centering 
    \includegraphics[width=4cm]{/logos/vscode.png} 
    \caption{Figura 4.11 - VSCode}
\end{figure}


\subsection{Database}
\subsection*{MySQL}
\begin{figure}[H] 
    \centering 
    \includegraphics[width=4cm]{/logos/Mysql_barattolo.png} 
    \caption{Figura 4.12 - MySQL}
\end{figure}


\section{Criteri di scelta della soluzione}
\label{sec:criteri-scelta-soluzione}


\section{Integrazione con l'ambiente esistente}
\label{sec:integrazione-ambiente-esistente}


%\section{Ciclo di vita del software}
%\label{sec:ciclo-vita-software}

%\section{Progettazione}
%\label{sec:progettazione}

%\subsubsection{Namespace 1} %**************************
%Descrizione namespace 1.

%\begin{namespacedesc}
    %\classdesc{Classe 1}{Descrizione classe 1}
    %\classdesc{Classe 2}{Descrizione classe 2}
%\end{namespacedesc}


%\section{Design Pattern utilizzati}

%\section{Codifica}
