\chapter{Tecnologie e principi teorici}
\label{cap:tecnologie-principi-teorici}

\intro{Il capitolo presenta il quadro teorico e tecnologico di riferimento del progetto, descrivendo i principali approcci e strumenti per il monitoraggio delle prestazioni applicative. Vengono illustrate le tecnologie analizzate e le motivazioni che hanno guidato la scelta della soluzione, in relazione ai requisiti e agli obiettivi del sistema di osservabilità sviluppato.}\\

\section{APM e Observability}
\label{sec:apm-observability}
L'osservabilità di un sistema si basa sull'analisi congiunta di tre pilastri fondamentali: metriche, \emph{log} e tracce. 
\begin{itemize}
    \item \textbf{Logs}: un \emph{log} è un \emph{record} testuale di un evento ad un orario specifico, come un tentativo di accesso, un errore di sistema o una transazione completata. I \emph{log} forniscono dettagli di contesto che aiutano a diagnosticare problemi specifici;
    \item \textbf{Metriche}: una metrica è una misurazione numerica di un aspetto specifico delle prestazioni del sistema, come l'utilizzo della \gls{cpug}, la latenza delle richieste o il numero di utenti attivi;
    \item \textbf{Tracce}: una traccia rappresenta il percorso di una singola richiesta attraverso i vari componenti di un sistema distribuito, consentendo di visualizzare il flusso delle operazioni e identificare i colli di bottiglia.
\end{itemize}
Un sistema \gls{apmg} moderno integra questi aspetti per fornire una visione completa dello stato e del comportamento di un applicazione. Nel contesto del monitoraggio di applicazioni \emph{web} come \emph{PetClinic}, l'osservabilità consente di analizzare i dati di \emph{performance} raccolti dal sistema di \gls{apmg} per identificare e risolvere problemi, ottimizzare le prestazioni e migliorare l'esperienza utente complessiva.


\section{Approcci e tecnologie per il monitoraggio}
\label{sec:approcci-tecnologie-monitoraggio}
Nel mondo dell'\emph{Application Performance Monitoring} esistono diversi approcci e tecnologie per raccogliere, analizzare e visualizzare i dati di telemetria. \\
Il monitoraggio delle prestazioni può essere realizzato mediante diverse strategie, che si distinguono per tipo di dati raccolti, modalità di raccolta e grado di integrazione con l'applicazione.


\subsection{Approcci principali}
Gli approcci al monitoraggio delle applicazioni possono essere classificati in base alla modalità di raccolta dei dati e al livello di osservabilità fornito. \\
Tra i principali si distinguono:
\begin{itemize}
\item \textbf{Monitoraggio basato sugli agenti (Agent-based monitoring)\footcite{article:agent-based-monitoring}:} prevede l'integrazione di componenti \emph{software}, detti \emph{agent}, all'interno dell'applicazione o dell'infrastruttura. Questi raccolgono metriche, \emph{log} e tracce in tempo reale, offrendo una visione complessiva del sistema. È il modello adottato da strumenti come \emph{Elastic APM};

\item \textbf{Monitoraggio senza agenti (Agentless monitoring)\footcite{article:agentless-monitoring}:} in questo caso la raccolta dei dati avviene tramite l'analisi di \emph{log} o metriche esposte da servizi esterni, senza modificare il codice dell'applicazione. Sebbene riduca l'invasività, questo approccio offre un livello di dettaglio inferiore rispetto ai sistemi \emph{agent-based};

\item \textbf{Distributed tracing\footcite{article:distributed-tracing}:} metodo che consente di tracciare l'intero ciclo di vita di una richiesta distribuita tra più servizi o microservizi, associando a ciascun evento un identificativo univoco;

\item \textbf{Synthetic monitoring\footcite{article:synthetic-monitoring}:} utilizza richieste simulate e \emph{test} automatizzati per verificare la disponibilità e le prestazioni delle applicazioni da diverse località geografiche. È utile per individuare problemi prima che impattino sugli utenti reali dell'applicazione;

\item \textbf{Real User Monitoring (RUM)\footcite{article:real-user-monitoring}:} misura le prestazioni dal punto di vista dell'utente reale, analizzando tempi di caricamento, interazioni e metriche di esperienza. Combinato con il monitoraggio sintetico, fornisce una visione completa della \emph{user experience} indicata con il termine \gls{eumg}\glsfirstoccur.
\end{itemize}
Nel corso del monitoraggio di \emph{PetClinic}, l'approccio adottato integra principalmente l'\emph{Agent-based monitoring}, il \emph{distributed tracing}, il \emph{Synthetic Monitoring} e il \emph{Real User Monitoring}, al fine di ottenere una visione dettagliata delle prestazioni dell'applicazione.


\subsection{Tecnologie e piattaforme note}
Negli ultimi anni si sono affermate diverse piattaforme e \emph{framework} dedicati al monitoraggio e all'osservabilità, che adottano architetture e modelli di raccolta dati differenti. \\
Tra le più rilevanti si trovano:
\begin{itemize}
\item \textbf{Elastic Stack:} una delle soluzioni \emph{open source} più diffuse, integra \emph{Elasticsearch}, \emph{Logstash} e \emph{Kibana}, estesa con \emph{Elastic APM} per il tracciamento delle prestazioni. Offre un ecosistema unificato per metriche, \emph{log} e tracce;

\item \textbf{OpenTelemetry:} standard \emph{open source} promosso dalla \emph{Cloud Native Computing Foundation (CNCF)} per la raccolta e l'esportazione di dati di telemetria. Fornisce \gls{sdkg}\glsfirstoccur e agenti per numerosi linguaggi di programmazione, garantendo interoperabilità tra sistemi di osservabilità differenti;

\item \textbf{Prometheus e Grafana:} soluzione \emph{open source} focalizzata sulle metriche. \emph{Prometheus} raccoglie e memorizza dati temporali, mentre \emph{Grafana} li visualizza in \emph{dashboard} personalizzabili. È molto usata in ambienti \emph{cloud-native} e \emph{Kubernetes};

\item \textbf{Datadog, New Relic, Dynatrace:} piattaforme commerciali che offrono funzionalità avanzate di \gls{apmg}, monitoraggio dell'infrastruttura e analisi basata su \emph{machine learning}.
\end{itemize}
Le soluzioni \emph{open source}, come \emph{Elastic Stack} e \emph{OpenTelemetry}, offrono maggiore flessibilità e possibilità di personalizzazione, rendendole particolarmente adatte per ambienti di sviluppo e sperimentazione. \\
Nel contesto del progetto di monitoraggio \emph{PetClinic}, l'attenzione si è concentrata sull'integrazione tra l'\emph{Elastic Stack} e \emph{OpenTelemetry}, con l'obiettivo di realizzare una soluzione di monitoraggio estendibile e compatibile con l'infrastruttura aziendale.


\section{Tecnologie e strumenti utilizzati}
\label{sec:tecnologie-strumenti}
Di seguito viene data una panoramica delle tecnologie e degli strumenti utilizzati.

\subsection{Elastic Stack}

\subsection*{Elasticsearch}
\emph{Elasticsearch}\footcite{site:elasticsearch} è un motore di ricerca e analisi distribuito basato su \emph{Apache Lucene}, progettato per gestire grandi volumi di dati in tempo reale. Fornisce funzionalità avanzate di ricerca \emph{full-text}, analisi dei dati e aggregazioni, rendendolo ideale per applicazioni di monitoraggio, analisi dei \emph{log} e \emph{business intelligence}. \\
\emph{Elasticsearch} supporta un'architettura scalabile e flessibile, consentendo di distribuire i dati su più nodi e \emph{cluster} per garantire alta disponibilità e prestazioni elevate.
\begin{figure}[H] 
    \centering 
    \includegraphics[width=4cm]{/logos/elasticsearch.png} 
    \caption{Logo Elasticsearch}
\end{figure}

\vspace{1em}

\subsection*{Kibana}
\emph{Kibana}\footcite{site:kibana} è uno strumento di visualizzazione e analisi dei dati \emph{open source}, progettato per lavorare in stretta integrazione con \emph{Elasticsearch}. Fornisce un'interfaccia utente intuitiva che consente agli utenti di creare \emph{dashboard} interattive, visualizzare grafici, mappe e tabelle, e analizzare i dati in tempo reale. \\
\emph{Kibana} supporta una vasta gamma di visualizzazioni personalizzabili e offre funzionalità avanzate come il filtraggio dei dati, la ricerca \emph{full-text} e l'esplorazione delle relazioni tra i dati, rendendolo uno strumento potente per l'analisi dei \emph{log}, il monitoraggio delle prestazioni e la \emph{business intelligence}.
\begin{figure}[H] 
    \centering 
    \includegraphics[width=4cm]{/logos/kibana.png} 
    \caption{Logo Kibana}
\end{figure}

\vspace{1em}

\subsection*{Elastic Agent e Fleet}
\emph{Elastic Agent}\footcite{site:elastic-agent} è un agente unificato sviluppato da \emph{Elastic} che consente di raccogliere, monitorare e proteggere i dati da diverse fonti in modo semplice ed efficace. \\
Integrando funzionalità di raccolta dati, sicurezza e monitoraggio, \emph{Elastic Agent} semplifica la gestione degli agenti e riduce la complessità operativa. \\
\emph{Fleet} è una funzionalità di gestione centralizzata all'interno di \emph{Kibana} che consente di distribuire, configurare e monitorare gli \emph{Elastic Agent} in modo scalabile e automatizzato. Attraverso \emph{Fleet}, gli utenti possono gestire facilmente le \emph{policy} di raccolta dati, aggiornare gli agenti e monitorare lo stato della loro infrastruttura da un'unica interfaccia.
\begin{figure}[H] 
    \centering 
    \includegraphics[width=\columnwidth]{fleet-agents.png} 
    \caption{Funzionamento Elastic Agents e Fleet}
    \label{fig:fleet-agents}
\end{figure}
La figura \ref{fig:fleet-agents} mostra l'architettura generale di gestione degli agenti tramite \emph{Fleet}. Gli \emph{Elastic Agents} installati sull'applicazione comunicano con il \emph{Fleet Server} per ricevere le \emph{policy} di configurazione e inviare i dati raccolti a \emph{Elasticsearch} per la visualizzazione in \emph{Kibana}.   
\vspace{1em}

\subsection*{OpenTelemetry}
\emph{OpenTelemetry}\footcite{site:opentelemetry} è un progetto \emph{open source} che fornisce un insieme di \gls{apig}\glsfirstoccur, librerie e strumenti per la raccolta di dati di telemetria da applicazioni e servizi. \\
L'obiettivo di \emph{OpenTelemetry} è standardizzare la raccolta e l'esportazione di dati di telemetria mediante l'\gls{otlpg}, facilitando l'integrazione con diversi sistemi di monitoraggio e analisi, tra cui l'\emph{Elastic Stack}. \\
\emph{OpenTelemetry} supporta vari linguaggi di programmazione e offre un'architettura che consente agli sviluppatori di personalizzare la raccolta dei dati in base alle esigenze specifiche delle loro applicazioni.
\begin{figure}[H] 
    \centering 
    \includegraphics[width=4cm]{/logos/OpenTelemetry.png} 
    \caption{Logo OpenTelemetry}
\end{figure}

\vspace{1em}

\subsection*{Elastic APM e APM Server}
\emph{Elastic APM}\footcite{site:apm} è una soluzione di monitoraggio delle prestazioni applicative sviluppata da \emph{Elastic}, progettata per tracciare e analizzare le prestazioni delle applicazioni in tempo reale. \\
\emph{Elastic APM} consente di identificare colli di bottiglia, errori e problemi di latenza, fornendo una visione dettagliata del comportamento dell'applicazione attraverso tracce distribuite, metriche e \emph{log}. \\
L'\emph{APM Server} è un componente dell'\emph{Elastic stack} che funge da punto di raccolta per i dati di telemetria inviati dagli agenti integrati nelle applicazioni. L'\emph{APM Server} elabora questi dati e li invia a \emph{Elasticsearch} per l'indicizzazione, consentendo agli utenti di visualizzare le informazioni tramite \emph{Kibana}.
\begin{figure}[H] 
    \centering 
    \includegraphics[width=4cm]{/logos/ElasticAPM.png} 
    \caption{Logo Elastic APM}
\end{figure}

\vspace{1em}

\subsection*{RUM Agent e Synthetic Monitoring}
Il \emph{RUM Agent} \emph{(Real User Monitoring Agent)}\footcite{site:rum-js}, nel contesto del progetto di monitoraggio di \emph{PetClinic} è una libreria \emph{JavaScript} che consente di monitorare le prestazioni dell'applicazione \emph{web} dal punto di vista dell'utente finale. \\
Integrando il \emph{RUM Agent} nelle pagine \emph{web}, è possibile raccogliere dati sulle interazioni degli utenti, i tempi di caricamento delle pagine e altri eventi significativi. Questi dati vengono inviati all'\emph{APM Server} per l'analisi e la visualizzazione tramite \emph{Kibana}. \\
Il \emph{Synthetic Monitoring}\footcite{site:synthetic}, invece, utilizza \emph{script} automatizzati per simulare le interazioni degli utenti con l'applicazione \emph{web}, consentendo di testare la disponibilità e le prestazioni da diverse località geografiche. \\
Nel progetto, tale monitoraggio è stato implementato tramite \emph{Kibana Synthetics}, che utilizza \emph{Playwright} come motore di esecuzione di \emph{script} basati su \emph{browser}. Sono stati quindi sviluppati \emph{script Playwright} dedicati, eseguiti da \emph{Kibana} per riprodurre scenari di utilizzo reali e verificare la corretta operatività dell'applicazione.
\begin{figure}[H] 
    \centering 
    \includegraphics[width=7cm]{/logos/playwright.png} 
    \caption{Logo Playwright}
\end{figure}


\newpage
\subsection{Linguaggi di programmazione}
\subsection*{Python}
\emph{Python}\footcite{site:python} è un linguaggio di programmazione ad alto livello, ampiamente utilizzato in vari ambiti, tra cui lo sviluppo \emph{web}, l'analisi dei dati, l'intelligenza artificiale e l'automazione. \\
\emph{Python} è dotato di una vasta libreria standard e di un ecosistema ricco di pacchetti e \emph{framework} che ne estendono le funzionalità e lo rendono un linguaggio versatile e potente. \\
In questo progetto \emph{Python} è stato utilizzato principalmente per la creazione di \emph{script} di automazione tramite \emph{Selenium}, al fine di eseguire test automatizzati sull'applicazione \emph{web PetClinic}.
\begin{figure}[H] 
    \centering 
    \includegraphics[width=4cm]{/logos/Python_logonormal.png} 
    \caption{Logo Python}
\end{figure}

\vspace{1em}

\subsection*{JavaScript}
\emph{JavaScript}\footcite{site:javascript} è un linguaggio di programmazione interpretato, utilizzato per lo sviluppo di applicazioni \emph{web} lato \emph{client} che consente di creare interfacce utente interattive e dinamiche. \\
Nel contesto del progetto di monitoraggio di \emph{PetClinic}, \emph{JavaScript} è stato utilizzato per integrare il \emph{RUM Agent}, permettendo la raccolta di dati sulle prestazioni dell'applicazione dal punto di vista dell'utente finale.
\begin{figure}[H] 
    \centering 
    \includegraphics[width=4cm]{/logos/javascript.png} 
    \caption{Logo JavaScript}
\end{figure}

\vspace{1em}

\subsection*{Java}
\emph{Java}\footcite{site:java} è un linguaggio di programmazione ad alto livello, orientato agli oggetti, ampiamente utilizzato nel mondo dello sviluppo \emph{software}. \\
\emph{OpenTelemetry} fornisce un agente \emph{APM} specifico per \emph{Java}, che consente di monitorare le prestazioni di \emph{backend} dell'applicazione \emph{PetClinic} in modo dettagliato integrando l'agente nel codice dell'applicazione.
\begin{figure}[H] 
    \centering 
    \includegraphics[width=4cm]{/logos/java.png} 
    \caption{Logo Java}
\end{figure}



\subsection{Framework e librerie}
\subsection*{Node.js}
\emph{Node.js}\footcite{site:nodejs} è utilizzato in \emph{PetClinic} lato \emph{server} per eseguire codice \emph{JavaScript}. Funge da intermediario per le richieste tra il \emph{client} e il \emph{server}, gestendo operazioni asincrone e migliorando le prestazioni complessive dell'applicazione \emph{web}.
\begin{figure}[H] 
    \centering 
    \includegraphics[width=4cm]{/logos/nodeJS.png} 
    \caption{Logo Node.js}
\end{figure}

\vspace{1em}

\subsection*{Spring Boot}
\emph{Spring Boot}\footcite{site:spring} è il \emph{framework} principale utilizzato per sviluppare l'applicazione \emph{PetClinic}. Fornisce un ambiente di sviluppo semplificato per la creazione di applicazioni \emph{Java} basate su \emph{Spring}, offrendo funzionalità integrate per la gestione delle dipendenze, la configurazione automatica e il supporto per vari moduli come \emph{Spring MVC}, \emph{Spring Data} e \emph{Spring Security}.
\begin{figure}[H] 
    \centering 
    \includegraphics[width=4cm]{/logos/spring.png} 
    \caption{Logo Spring}
\end{figure}

\vspace{1em}

\subsection*{Selenium}
\emph{Selenium}\footcite{site:selenium} è un \emph{framework open source} utilizzato per l'automazione dei \emph{test} delle applicazioni \emph{web}. Consente di simulare le interazioni degli utenti con il \emph{browser}, eseguendo \emph{test} funzionali e di regressione in modo automatizzato. \\
\emph{Selenium} supporta diversi linguaggi di programmazione, tra cui \emph{Java}, \emph{Python} e \emph{JavaScript} e può essere integrato con vari strumenti di \emph{testing} e \emph{framework} di sviluppo. \\
Durante il progetto è stata utilizzata la libreria \emph{Selenium} per \emph{Python} per creare \emph{script} di \emph{test} automatizzati che simulano le azioni degli utenti sull'applicazione \emph{PetClinic}.
\begin{figure}[H] 
    \centering 
    \includegraphics[width=4cm]{/logos/selenium.png} 
    \caption{Logo Selenium}
\end{figure}



\subsection{Strumenti di sviluppo}
\subsection*{VSCode}
\emph{Visual Studio Code (VSCode)}\footcite{site:vscode} è un \emph{editor} di codice sorgente sviluppato da \emph{Microsoft}. Supporta una vasta gamma di linguaggi di programmazione e offre diverse funzionalità avanzate come il \emph{debugging} integrato, il controllo della versione tramite \emph{Git} e un \emph{marketplace} ricco di estensioni.
\begin{figure}[H] 
    \centering 
    \includegraphics[width=4cm]{/logos/vscode.png} 
    \caption{Logo VSCode}
\end{figure}



\subsection{Database}
\subsection*{MySQL}
\emph{MySQL}\footcite{site:mysql} è un sistema di gestione di \emph{database} relazionali \emph{open source}, ampiamente utilizzato per la memorizzazione e la gestione dei dati in applicazioni \emph{web} e aziendali. \emph{MySQL} supporta il linguaggio \gls{sqlg}\glsfirstoccur per l'interrogazione e la manipolazione dei dati, offrendo funzionalità avanzate come transazioni, indicizzazione e replicazione. \\
Nel contesto dell'applicazione \emph{PetClinic}, \emph{MySQL} viene utilizzato come \emph{database} per archiviare le informazioni relative agli utenti, agli animali domestici, alle visite veterinarie e ad altri dati pertinenti all'applicazione.
\begin{figure}[H] 
    \centering 
    \includegraphics[width=4cm]{/logos/Mysql_barattolo.png} 
    \caption{Logo MySQL}
\end{figure}


\subsection{Containerizzazione}
\subsection*{Docker}
\emph{Docker}\footcite{site:docker} è una piattaforma di containerizzazione che consente di automatizzare il \emph{deployment} di applicazioni all'interno di \emph{container} leggeri e portabili. I \emph{container} isolano le applicazioni e le loro dipendenze, garantendo coerenza tra gli ambienti di sviluppo, \emph{test} e produzione. \\
Nel contesto dell'applicazione \emph{PetClinic}, \emph{Docker} è stato utilizzato per creare un ambiente di sviluppo replicabile e per semplificare il \emph{deployment} dell'applicazione e dei suoi componenti.
\begin{figure}[H] 
    \centering 
    \includegraphics[width=4cm]{/logos/docker.png} 
    \caption{Logo Docker}
\end{figure}


\subsection*{Docker Compose}
\emph{Docker Compose}\footcite{site:docker-compose} è uno strumento che consente di definire e gestire applicazioni \emph{multi-container} tramite un file di configurazione \texttt{docker-compose.yml}. \\
Attraverso \emph{Compose} è possibile avviare, fermare e orchestrare più servizi correlati come un'unica applicazione, semplificando la gestione degli ambienti di sviluppo e test. \\
Nel contesto dell'applicazione \emph{PetClinic}, \emph{Docker Compose} è stato utilizzato per avviare simultaneamente i componenti dell'applicazione, come il servizio principale, il \emph{database} e gli strumenti di monitoraggio.
\begin{figure}[H] 
    \centering 
    \includegraphics[width=4cm]{/logos/docker.png} 
    \caption{Logo Docker Compose}
\end{figure}


\subsection{Tecnologie analizzate}
\subsection*{Model Context Protocol (MCP)}
Durante lo \emph{stage} è stato analizzato il \emph{Model Context Protocol}\footcite{site:mcp} (\gls{mcpg}\glsfirstoccur), uno standard progettato per integrare modelli di intelligenza artificiale con strumenti esterni. \\
Il protocollo permette di esporre funzionalità tramite un \emph{server MCP} che può essere utilizzato da applicazioni \gls{aig} (come agenti conversazionali o strumenti di automazione). \\
Sebbene le funzionalità di \gls{aig} dello \emph{stack Elastic} come l'\emph{AI Assistant} o gli \emph{Elastic AI Agents} siano pienamente disponibili solo a partire dalla versione 9.2, è stata comunque analizzata la configurazione del \emph{server MCP} con la versione 8.15.3, preparando una procedura per il suo avvio e integrazione preliminare.
\begin{figure}[H] 
    \centering 
    \includegraphics[width=\columnwidth]{mcp-simple-diagram.png} 
    \caption{Protocollo MCP}
\end{figure}



\section{Criteri di scelta della soluzione}
\label{sec:criteri-scelta-soluzione}
Le principali motivazioni che hanno guidato la scelta della soluzione di monitoraggio basata su \emph{Elastic Stack} e \emph{OpenTelemetry} sono:
\begin{itemize}
    \item \textbf{Open source e flessibilità:} entrambe le tecnologie sono \emph{open source}, consentendo una maggiore personalizzazione e adattabilità alle esigenze specifiche del progetto;
    \item \textbf{Scalabilità:} la soluzione è progettata per gestire grandi volumi di dati in ambienti distribuiti, garantendo prestazioni elevate anche con l'aumento del carico di lavoro;
    \item \textbf{Ecosistema completo:} \emph{Elastic Stack} offre un ecosistema completo per la raccolta, l'analisi e la visualizzazione dei dati, semplificando la gestione del sistema di monitoraggio;
    \item \textbf{Requisiti aziendali:} la scelta è stata influenzata dalla necessità di integrare il sistema di monitoraggio con l'infrastruttura esistente dell'azienda, che già utilizza componenti dell'\emph{Elastic Stack} e \emph{Docker};
    \item \textbf{Comunità attiva:} entrambe le tecnologie vantano una comunità di sviluppatori attiva e in crescita, che contribuisce al miglioramento continuo delle piattaforme e fornisce supporto agli utenti.
\end{itemize}


\section{Integrazione con l'ambiente esistente}
\label{sec:integrazione-ambiente-esistente}
L'integrazione della soluzione di monitoraggio è stata progettata tenendo conto delle specifiche dell'ambiente tecnico aziendale, basato su infrastrutture \emph{Linux} e containerizzazione tramite \emph{Docker}. \\
Tutti i componenti dell'\emph{Elastic Stack} sono stati distribuiti in un ambiente isolato, gestito tramite \emph{Docker}, mantenendo la compatibilità con le versioni approvate dall'azienda. \\
La comunicazione tra l'applicazione \emph{web PetClinic} e il sistema di osservabilità avviene tramite l'agente \emph{OpenTelemetry Java}, che esporta i dati di telemetria verso l'\emph{Elastic APM Server} gestito da \emph{Fleet} dopo essere passati per un \emph{Collector OpenTelemetry}. \\
L'approccio adottato permette un'integrazione trasparente con l'ambiente esistente, garantendo la scalabilità del sistema e la possibilità di estendere la soluzione ad altri servizi o applicazioni monitorate nel futuro.

%\section{Ciclo di vita del software}
%\label{sec:ciclo-vita-software}

%\section{Progettazione}
%\label{sec:progettazione}

%\subsubsection{Namespace 1} %**************************
%Descrizione namespace 1.

%\begin{namespacedesc}
    %\classdesc{Classe 1}{Descrizione classe 1}
    %\classdesc{Classe 2}{Descrizione classe 2}
%\end{namespacedesc}


%\section{Design Pattern utilizzati}

%\section{Codifica}
