\chapter{Tecnologie e principi teorici}
\label{cap:tecnologie-principi-teorici}

\intro{Il capitolo presenta il quadro teorico e tecnologico di riferimento del progetto, descrivendo i principali approcci e strumenti per il monitoraggio delle prestazioni applicative. Vengono illustrate le tecnologie analizzate e le motivazioni che hanno guidato la scelta della soluzione, in relazione ai requisiti e agli obiettivi del sistema di osservabilità sviluppato.}\\

\section{APM e Observability}
\label{sec:apm-observability}
Definizione, Obiettivi e componenti tipici di un sistema APM.


\section{Approcci e tecnologie per il monitoraggio}
\label{sec:approcci-tecnologie-monitoraggio}

\subsection{Approcci principali}


\subsection{Tecnologie e piattaforme note}


\section{Tecnologie e strumenti}
\label{sec:tecnologie-strumenti}
Di seguito viene data una panoramica delle tecnologie e strumenti utilizzati.

\subsection*{Elasticsearch}

\subsection*{Logstash}

\subsection*{Kibana}

\subsection*{Beats}

\subsection*{Elastic Agent e Fleet}

\subsection*{OpenTelemetry}

\subsection*{Elastic APM e APM Server}

\subsection*{RUM Agent e Synthetic Monitoring}

\subsection*{MCP}


\section{Criteri di scelta della soluzione}
\label{sec:criteri-scelta-soluzione}


\section{Integrazione con l'ambiente esistente}
\label{sec:integrazione-ambiente-esistente}


%\section{Ciclo di vita del software}
%\label{sec:ciclo-vita-software}

%\section{Progettazione}
%\label{sec:progettazione}

%\subsubsection{Namespace 1} %**************************
%Descrizione namespace 1.

%\begin{namespacedesc}
    %\classdesc{Classe 1}{Descrizione classe 1}
    %\classdesc{Classe 2}{Descrizione classe 2}
%\end{namespacedesc}


%\section{Design Pattern utilizzati}

%\section{Codifica}
