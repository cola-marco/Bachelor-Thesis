\chapter{Sviluppo del prodotto}
\label{cap:product-development}

\section{Architettura complessiva del sistema}
L'architettura realizzata durante il tirocinio ha l'obiettivo di fornire un sistema completo di osservabilità per la \emph{web app PetClinic}, integrando in un ambiente unico la raccolta dei log, metriche, tracce e monitoraggio sintetico. \\
Per raggiungere questo obiettivo è stato utilizzato l'ecosistema \emph{Elastic Stack}, \emph{Docker} e \emph{OpenTelemetry}. \\
L'applicazione PetClinic è strumentata tramite l'\emph{OpenTelemetry Java Agent}, che esporta metriche e tracce verso un \emph{OpenTelemetry Collector}. Quest'ultimo funge da punto di aggregazione e inoltra i dati al sistema \gls{apm} gestito da \emph{Elastic Agent} tramite \emph{Fleet}. \\
In parallelo, i log dell'applicazione vengono scritti su un file locale dal \emph{container} \emph{PetClinic} e successivamente raccolti da \emph{Logstash}, che li elabora e li inoltra verso Elasticsearch seguendo una \emph{pipeline} personalizzata. \\
Per quanto riguarda la visualizzazione dei dati, è stato utilizzato \emph{Kibana}, tramite cui è possibile monitorare l'andamento dell'applicazione, creare dashboard, analizzare metriche di performance, effettuare ricerche sui log ed eseguire attività di anomaly detection. \\
Accanto ai dati reali provenienti dall'applicazione, è stato integrato un sistema di \emph{Synthetic Monitoring} basato su \emph{Kibana Synthetics}, che utilizza script \emph{Playwright} per simulare il comportamento degli utenti e verificare la disponibilità e il corretto funzionamento dei principali flussi di navigazione. Trattandosi di un applicazione di prova infatti l'utilizzo di monitoraggio sintetico permette di lavorare con una quantità più significativa di dati. \\
Nel complesso, l'architettura si presenta come una \emph{pipeline} altamente modulare, in cui ogni componente è orchestrato tramite \emph{Docker Compose}. Questa struttura rende l'ambiente facilmente replicabile, facilitando sia le attività di sviluppo che quelle di \emph{troubleshooting}.
\begin{figure}[H] 
    \centering 
    \includegraphics[width=\columnwidth]{Architettura_complessiva.png}
    \caption{Figura 5.1 - Architettura complessiva}
\end{figure}



\section{Struttura del progetto e organizzazione del repository}
La realizzazione del sistema di osservabilità è stata accompagnata dalla definizione di una struttura del progetto che ne facilitasse sia lo sviluppo che le attività di manutenzione. \\
Il repository principale, chiamato \emph{elastic-project}, contiene tutte le componenti necessarie all'avvio dell'infrastruttura tramite \emph{Docker Compose} e alla configurazione delle \emph{pipelines} di log, metriche e tracce, alla configurazione delle \emph{pipeline} di log, metriche e tracce, nonché all'avvio delle istanze del \emph{Fleet Server} e delle \emph{policy} per il \emph{Synthetic monitoring}. \\
La struttura complessiva è riportata di seguito:
\begin{lstlisting}[basicstyle=\ttfamily\small]
- elastic-project/ 
-- docker-compose.yml
-- .env
-- logstash.conf
-- logs/
-- collector/
    -> config.yaml
-- mcp-server-elasticsearch/
-- spring-petclinic/
    -> Dockerfile
\end{lstlisting}

Questa organizzazione è stata costruita con l'obiettivo di isolare le responsabilità dei vari componenti:
\begin{itemize}
    \item \textbf{docker-compose.yml} rappresenta il file principale per l'orchestrazione dei \emph{container}, in cui vengono definiti i servizi fondamentali. La scelta di accorpare tutti i servizi in un unico file semplifica la fase di avvio e garantisce un ambiente riproducibile;
    \item \textbf{spring-petclinic/} contiene il codice dell'applicazione \emph{PetClinic} e il relativo \emph{Dockerfile} per la creazione dell'immagine personalizzata con l'\emph{OpenTelemetry Java Agent} integrato;
    \item \textbf{collector/} include la configurazione dell'\emph{OpenTelemetry Collector} nel file \emph{config.yaml}, responsabile della ricezione dei dati \gls{otlp} provenienti da \emph{PetClinic} e del loro inoltro al sistema \gls{apm} gestito da \emph{Fleet};
    \item \textbf{logstash.conf} definisce la \emph{pipeline} di \emph{Logstash} che legge e filtra i log dell'applicazione, applicando un primo livello di arricchimento e inviando i dati a \emph{Elasticsearch};
    \item \textbf{logs/} funge da volume locale in cui l'applicazione \emph{PetClinic} rende disponibili i file di log;
    \item \textbf{mcp-server-elasticsearch/} contiene i file necessari per l'esecuzione del \emph{server} \gls{mcpg}\glsfirstoccur dedicato a \emph{Elasticsearch}, che abilita l'interazione con strumenti di \gls{ai} generativa come \emph{Claude Code};
    \item \textbf{.env} contiene le variabili d'ambiente necessarie per la configurazione dei servizi.
\end{itemize}
Questa struttura ha permesso di lavorare in modo indipendente sulle singole componenti del sistema senza introdurre interferenze tra servizi, e ha contribuito a semplificare la fase di diagnosi dei problemi.

\newpage


\section{Implementazione del logging}



\section{Implementazione di traces e metrics con OpenTelemetry}



\section{Containerizzazione e orchestrazione con Docker Compose}



\section{Visualizzazione dei dati in Kibana}



\section{Creazione delle dashboard}



\subsection{Dashboard Frontend - User Journey}



\subsection{Dashboard Backend - Health \& Stability}



\section{Machine Learning e Anomaly Detection}



\section{Synthetic Monitoring mediante Playwright e Kibana Synthetics}



\section{Integrazione con MCP Server e strumenti AI}



\section{Sintesi dei flussi end-to-end implementati}