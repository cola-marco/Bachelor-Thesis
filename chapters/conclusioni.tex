\chapter{Conclusioni}
\label{cap:conclusioni}

\section{Consuntivo finale}
Il progetto ha portato alla realizzazione di una piattaforma completa di osservabilità per la \emph{web application PetClinic}, comprendente la raccolta e l'analisi di \emph{log}, metriche, tracce distribuite e \emph{test} sintetici. È stata inoltre configurata un'infrastruttura basata su \emph{Docker Compose}, integrata con \emph{Fleet}, \emph{Logstash} e l'\emph{OpenTelemetry Collector}, affiancata da strumenti di analisi avanzati quali \emph{Machine Learning} e \emph{Anomaly Detection}. \\
Al termine dello \emph{stage}, l'intera soluzione risulta funzionante, validata ed avviabile tramite un unico comando, garantendo riproducibilità e coerenza dell'ambiente. \\
L'intera piattaforma inoltre è compatibile con le versioni utilizzate in azienda dello \emph{stack Elastic}, facilitando un eventuale integrazione futura. \\
Durante le ultime settimane di \emph{stage} sono stati condotti più \emph{test} sulla piattaforma alla presenza dei membri del gruppo, al fine di allineare tutti i presenti sulle funzionalità implementate e raccogliere \emph{feedback} utili. \\
Al termine di questi \emph{test}, la piattaforma è stata replicata in un ambiente di laboratorio aziendale, per valutarne il funzionamento in un contesto reale. \\
Alla fine del percorso di \emph{stage} è stata inoltre resa disponibile una documentazione completa, ad illustrare il funzionamento di tutte le componenti della piattaforma, i pro e i contro delle tecnologie utilizzate e delle scelte effettuate, assieme alle possibili evoluzioni future. \\
In conclusione, il progetto ha raggiunto gli obiettivi prefissati, fornendo una soluzione efficace e ben documentata per l'osservabilità della \emph{web app PetClinic}.


\section{Valutazione dei rischi individuati}
A conclusione del progetto è stata effettuata una revisione dei rischi identificati nella fase iniziale (Sezione ~\ref{cap:analisi-rischi}) al fine di valutare quali si siano effettivamente presentati, quali siano stati mitigati e quali non abbiano avuto impatto sullo svolgimento delle attività. \\
In particolare:
\begin{itemize}
    \item \textbf{Inesperienza nella suite Elastic:}
        si è verificato in modo parziale nelle prime settimane, ma è stato mitigato tramite studio autonomo e supporto del \emph{tutor} aziendale.

    \item \textbf{Integrazione tra componenti Elastic:}  
        il rischio si è verificato, nello specifico sono stati riscontrati alcuni problemi di comunicazione tra \emph{Logstash}, \emph{APM Server} e \emph{Elasticsearch}, ma sono stati risolti tramite \emph{debug} e lettura dei \emph{log}.

    \item \textbf{Qualità e coerenza dei dati raccolti:}  
        il rischio non si è manifestato.

    \item \textbf{Scalabilità e carico del sistema:}  
        il rischio si è manifestato solamente all'interno dell'ambiente locale \gls{wslg} \emph{Linux} a causa di una impostazione errata di memoria al container di \emph{Elasticsearch}, ma è stato risolto aumentando la memoria allocata.

    \item \textbf{Implementazione di algoritmi di Machine Learning:}  
        alcuni limiti di licenza hanno inizialmente rallentato l'adozione del modulo \gls{mlg}, successivamente superati tramite l'attivazione del periodo di prova per i \emph{test} locali.

    \item \textbf{Problemi di configurazione delle pipeline Logstash:}  
        il rischio non si è manifestato.

    \item \textbf{Accesso limitato a funzionalità premium di Elastic:}  
        il rischio si è verificato nelle prime settimane; è stato poi mitigato con l'attivazione di un periodo di prova gratuito in locale.
\end{itemize}
Nel complesso, i rischi previsti si sono dimostrati utili a guidare il lavoro: quelli più rilevanti si sono presentati nella fase iniziale ma sono stati risolti, mentre altri non hanno avuto impatto significativo sul progetto.

\newpage

\section{Raggiungimento degli obiettivi}
Per completezza, si riporta la validazione dei requisiti individuati durante l'analisi descritta al capitolo \ref{cap:analisi-requisiti}, divisi in funzionali, non funzionali, qualitativi e di vincolo. \\  
Per ciascun requisito è stato indicato l'esito finale ottenuto al termine del progetto: soddisfatto, parzialmente soddisfatto oppure non applicabile.

\subsection{Raggiungimento dei requisiti funzionali}
I requisiti funzionali descrivono cosa deve fare il sistema.
Sono le funzionalità concrete che la soluzione deve offrire per raggiungere gli obiettivi del progetto.

\setlength{\LTcapwidth}{\textwidth}
\begin{center}
\begin{longtable}{|p{1.3cm}|p{7cm}|p{2.5cm}|p{1.7cm}|}
\caption{Tabella del tracciamento dei requisiti funzionali con esito}
\label{tab:requisiti-funzionali-esito}
\\
\hline
\rowcolor[gray]{0.8}
\textbf{Codice} & \textbf{Descrizione} & \textbf{Classificazione} & \textbf{Esito} \\
\hline
\endfirsthead

\hline
\rowcolor[gray]{0.8}
\textbf{Codice} & \textbf{Descrizione} & \textbf{Classificazione} & \textbf{Esito} \\
\hline
\endhead

\hline
\endfoot

\hline
\endlastfoot

ROF-1 & Il sistema deve permettere la raccolta automatica di metriche e \emph{log} relativi alla \emph{web application} tramite agenti \emph{OpenTelemetry} o \emph{Elastic APM}. & Obbligatorio & Soddisfatto \\
\hline

ROF-2 & Il sistema deve inviare i dati raccolti agli \emph{endpoint} \emph{Elasticsearch} per l'analisi e l'indicizzazione. & Obbligatorio & Soddisfatto \\
\hline

ROF-3 & Il sistema deve consentire la creazione di \emph{pipeline} di \emph{log} tramite \emph{Logstash} per filtraggio, trasformazione e inoltro dei dati in \emph{Elasticsearch}. & Obbligatorio & Soddisfatto \\
\hline

ROF-4 & Il sistema deve prevedere la configurazione di \emph{Elastic Agents} (\emph{Beats}/\emph{\gls{apm}}) per la raccolta dati della navigazione. & Obbligatorio & Soddisfatto \\
\hline

ROF-5 & Il sistema deve generare \emph{dashboard} avanzate e visualizzazioni in \emph{Kibana}, con metriche di \emph{performance}, accesso e flussi utente. & Obbligatorio & Soddisfatto \\
\hline

ROF-6 & Il sistema deve permettere la verifica della corretta acquisizione dei dati e la loro indicizzazione in \emph{Elasticsearch}. & Obbligatorio & Soddisfatto \\
\hline

ROF-7 & Il sistema deve prevedere lo sviluppo e l'esecuzione di \emph{script} automatizzati in \emph{Python} o \emph{Java} per la simulazione del traffico utente (\emph{Synthetic Monitoring}). & Obbligatorio & Soddisfatto \\
\hline

ROF-8 & Deve essere possibile filtrare e ricercare i \emph{log} per \emph{host}, servizio, livello di severità o periodo temporale. & Obbligatorio & Soddisfatto \\
\hline

RDF-9 & Il sistema dovrebbe prevedere la configurazione di regole di \emph{alerting} e notifiche in tempo reale per anomalie rilevate. & Desiderabile & Soddisfatto \\
\hline

RDF-10 & Il sistema dovrebbe integrare algoritmi di \emph{Machine Learning} per l'individuazione automatica di anomalie. & Desiderabile & Soddisfatto \\
\hline

RDF-11 & Il sistema dovrebbe consentire l'esportazione delle \emph{dashboard} o dei risultati delle \emph{query} in formato \gls{pdfg} o \gls{csvg}. & Desiderabile & Soddisfatto \\
\hline

RZF-12 & Il sistema può prevedere un modulo aggiuntivo per la generazione automatica di report periodici delle metriche raccolte. & Opzionale & Parzialmente soddisfatto \\
\hline

RZF-13 & Il sistema può consentire l'importazione automatica delle configurazioni \gls{apm} da ambienti di \emph{test} o \emph{staging}. & Opzionale & Soddisfatto \\
\hline

\end{longtable}
\end{center}


\newpage
\subsection{Raggiungimento dei requisiti non funzionali}
I requisiti non funzionali definiscono come il sistema deve comportarsi, cioè le sue proprietà di qualità interna.
Non aggiungono nuove funzioni, ma impongono vincoli di prestazioni, sicurezza, disponibilità, scalabilità, affidabilità e manutenibilità.

\setlength{\LTcapwidth}{\textwidth}
\begin{center}
\begin{longtable}{|p{1.3cm}|p{7cm}|p{2.5cm}|p{1.7cm}|}
\caption{Tabella del tracciamento dei requisiti non funzionali con esito}
\label{tab:requisiti-non-funzionali-esito}
\\
\hline
\rowcolor[gray]{0.8}
\textbf{Codice} & \textbf{Descrizione} & \textbf{Classificazione} & \textbf{Esito} \\
\hline
\endfirsthead

\hline
\rowcolor[gray]{0.8}
\textbf{Codice} & \textbf{Descrizione} & \textbf{Classificazione} & \textbf{Esito} \\
\hline
\endhead

\hline
\endfoot

\hline
\endlastfoot

RON-1 & Il sistema deve essere scalabile e consentire l'aggiunta di nuove fonti di dati o agenti senza compromettere la stabilità. & Obbligatorio & Soddisfatto \\
\hline

RON-2 & Il sistema deve garantire l'affidabilità nella trasmissione e nella conservazione dei dati raccolti. & Obbligatorio & Soddisfatto \\
\hline

RON-3 & La piattaforma deve assicurare un tempo di latenza accettabile nella visualizzazione dei dati (< 5 secondi per l'aggiornamento delle \emph{dashboard}). & Obbligatorio & Soddisfatto \\
\hline

RDN-4 & Il sistema dovrebbe garantire la possibilità di eseguire \emph{test} di carico e stress per valutare la stabilità dell'ambiente. & Desiderabile & Non applicabile \\
\hline

RDN-5 & Il sistema dovrebbe supportare l'autenticazione per la gestione degli accessi a \emph{Kibana}. & Desiderabile & Soddisfatto \\
\hline

\end{longtable}
\end{center}


\newpage
\subsection{Raggiungimento dei requisiti qualitativi}
I requisiti qualitativi specificano le proprietà qualitative che influenzano l'esperienza d'uso e la manutenibilità.
Si concentrano su aspetti percepibili, come semplicità, chiarezza, flessibilità o estendibilità.

\setlength{\LTcapwidth}{\textwidth}
\begin{center}
\begin{longtable}{|p{1.3cm}|p{7cm}|p{2.5cm}|p{1.7cm}|}
\caption{Tabella del tracciamento dei requisiti qualitativi con esito}
\label{tab:requisiti-qualitativi-esito}
\\
\hline
\rowcolor[gray]{0.8}
\textbf{Codice} & \textbf{Descrizione} & \textbf{Classificazione} & \textbf{Esito} \\
\hline
\endfirsthead

\hline
\rowcolor[gray]{0.8}
\textbf{Codice} & \textbf{Descrizione} & \textbf{Classificazione} & \textbf{Esito} \\
\hline
\endhead

\hline
\endfoot

\hline
\endlastfoot

ROQ-1 & L'interfaccia di \emph{Kibana} deve offrire una rappresentazione chiara e intuitiva delle metriche principali. & Obbligatorio & Soddisfatto \\
\hline

ROQ-2 & I dati devono essere visualizzabili in forma aggregata e filtrabile in base a intervalli temporali e categorie di evento. & Obbligatorio & Soddisfatto \\
\hline

ROQ-3 & Le \emph{dashboard} devono presentare una chiara distinzione cromatica tra metriche positive, neutre e anomale. & Obbligatorio & Soddisfatto \\
\hline

RDQ-4 & Le \emph{dashboard} dovrebbero essere personalizzabili dall'utente secondo criteri di interesse (\emph{performance}, accessi, flussi). & Desiderabile & Soddisfatto \\
\hline

RZQ-5 & Il sistema può includere un \emph{layout dark/light mode} o temi grafici personalizzati per una migliore leggibilità. & Opzionale & Soddisfatto \\
\hline

\end{longtable}
\end{center}

\newpage
\subsection{Raggiungimento dei requisiti di vincolo}

Impongono limitazioni o condizioni esterne al progetto: ambienti, tecnologie,
compatibilità, strumenti, standard aziendali o legali.

\setlength{\LTcapwidth}{\textwidth}
\begin{center}
\begin{longtable}{|p{1.3cm}|p{7cm}|p{2.5cm}|p{1.7cm}|}
\caption{Tabella del tracciamento dei requisiti di vincolo con esito}
\label{tab:requisiti-vincolo-esito}
\\
\hline
\rowcolor[gray]{0.8}
\textbf{Codice} & \textbf{Descrizione} & \textbf{Classificazione} & \textbf{Esito} \\
\hline
\endfirsthead

\hline
\rowcolor[gray]{0.8}
\textbf{Codice} & \textbf{Descrizione} & \textbf{Classificazione} & \textbf{Esito} \\
\hline
\endhead

\hline
\endfoot

\hline
\endlastfoot

ROV-1 & Il sistema deve utilizzare i prodotti della suite \emph{Elastic Stack} (\emph{Elasticsearch, Logstash, Kibana, Beats, APM Server/Agent}). & Obbligatorio & Soddisfatto \\
\hline

ROV-2 & L'ambiente operativo deve essere \emph{Linux} (\emph{Redphat} o distribuzioni equivalenti). & Obbligatorio & Soddisfatto \\
\hline

ROV-3 & Tutti i componenti \emph{software} devono essere compatibili con la versione di \emph{Linux} installata (es. \emph{Ubuntu 22.04 LTS} o \emph{Red Hat 9}). & Obbligatorio & Soddisfatto \\
\hline

ROV-4 & Le componenti devono rispettare le seguenti versioni minime:
\begin{itemize}
    \item \emph{Python} >= 3.10
    \item \emph{Java} >= 16
    \item \emph{Node.js} >= 17
    \item \emph{Logstash} >= 8.10
    \item \emph{Kibana} >= 8.10
    \item \emph{Beats} >= 8.10
    \item \emph{APM Server} >= 8.10
    \item \emph{Elasticsearch} >= 8.10
    \item \emph{OpenTelemetry Java Agent} >= 1.26.0
\end{itemize} & Obbligatorio &Soddisfatto \\
\hline

ROV-5 & La \emph{web application} deve essere compatibile con i principali \emph{browser} (\emph{Chrome} >= 120, \emph{Firefox} >= 115, \emph{Edge} >= 120, \emph{Safari} >= 15). & Obbligatorio & Soddisfatto \\
\hline

ROV-6 & Le configurazioni devono essere eseguite in ambiente \emph{Docker} o su infrastruttura aziendale. & Obbligatorio & Soddisfatto \\
\hline

RDV-7 & La documentazione tecnica deve essere redatta in \emph{Markdown}, \emph{LaTeX} o \gls{pdfg}. & Desiderabile & Soddisfatto \\
\hline

RDV-8 & Il sistema dovrebbe supportare la distribuzione tramite \emph{Docker Compose}. & Desiderabile & Soddisfatto \\
\hline

\end{longtable}
\end{center}


\newpage
\section{Riepilogo dei requisiti raggiunti}

\begin{center}
\captionof{table}{Riepilogo dei requisiti raggiunti}
\label{tab:requisiti-riepilogo-finale}
\begin{tabular}{|l|c|c|c|c|}
\hline
\rowcolor[gray]{0.8}
\textbf{Tipologia} & \textbf{Obbligatorio} & \textbf{Desiderabile} & \textbf{Opzionale} & \textbf{Totale} \\
\hline
\textbf{Funzionali} & 8 & 3 & 2 & 13 \\
\hline
\textbf{Non funzionali} & 3 & 2 & 0 & 5 \\
\hline
\textbf{Qualitativi} & 3 & 1 & 1 & 5 \\
\hline
\textbf{Di vincolo} & 6 & 2 & 0 & 8 \\
\hline
\textbf{Totale} & 20 & 8 & 3 & 31 \\
\hline
\end{tabular}
\end{center}
Dall'analisi conclusiva emerge che il progetto ha soddisfatto la totalità dei requisiti classificati come \emph{obbligatori} e \emph{desiderabili}, per un totale di 28 requisiti su 31.  
I tre requisiti classificati come \emph{opzionali} sono stati valutati come non applicabili nell'ambito dello stage, in quanto non rientravano negli obiettivi tecnici concordati con l'azienda.  
Nel dettaglio, sono stati soddisfatti:
\begin{itemize}
    \item 8 requisiti funzionali obbligatori, 3 desiderabili e 1 opzionali;
    \item 3 requisiti non funzionali obbligatori e 1 desiderabile;
    \item 3 requisiti qualitativi obbligatori, 1 desiderabile e 1 opzionale;
    \item 6 requisiti di vincolo obbligatori e 2 desiderabili.
\end{itemize}
Complessivamente, il sistema realizzato rispetta pienamente le richieste principali definite nella fase di analisi dei requisiti.




\section{Conoscenze acquisite}
Lo sviluppo del progetto mi ha permesso di sviluppare competenze avanzate nell'ambito dell'osservabilità delle applicazioni \emph{web}, comprendendo l'utilizzo di tecnologie quali \emph{Elasticsearch}, \emph{Kibana}, \emph{Fleet}, \emph{Logstash} e l'\emph{OpenTelemetry Collector}. \\
Ho acquisito familiarità con la progettazione di \emph{pipeline} di dati, l'analisi delle metriche applicative, la creazione di \emph{dashboard}, l'indicizzazione tramite \emph{Elasticsearch} e la gestione del monitoraggio sintetico. \\
Sul piano operativo ho migliorato la capacità di lavorare in ambienti containerizzati, diagnosticare problemi distribuiti e strutturare documentazione tecnica chiara e replicabile. \\
L'opportunità di sviluppare in autonomia una soluzione completa mi ha permesso di affinare le competenze di \emph{problem solving}, gestione del tempo e comunicazione tecnica. \\
In sintesi, il progetto ha rappresentato un'importante occasione di crescita professionale, fornendomi importanti competenze pratiche e teoriche nel campo dell'osservabilità delle applicazioni moderne.



\section{Valutazione personale}
Il progetto di \emph{stage} ha rappresentato per me un'importante opportunità di crescita. Sono molto grato al team di Kirey per avermi supportato in ogni fase del percorso e per avermi seguito con attenzione e competenza. \\
Lavorare su un progetto come questo è stato stimolante in quanto mi è stata data carta bianca su come strutturare la piattaforma, permettendomi di sperimentare diverse soluzioni prima di arrivare alla soluzione definitiva con \emph{docker compose}. \\
Il lavoro svolto su questo progetto mi ha portato sicuramente a migliorare le mie competenze tecniche, in particolare nell'ambito dell'osservabilità delle applicazioni \emph{web} e della containerizzazione. \\
Inoltre, ho avuto l'opportunità di sviluppare competenze trasversali come la gestione del tempo, la comunicazione tecnica e il \emph{problem solving}, grazie alle riunioni con il \emph{team} e ai momenti di allineamento. \\
In conclusione, ritengo che questo progetto di \emph{stage} sia stato un'esperienza molto positiva e formativa, che mi ha permesso di acquisire nuove competenze e di crescere professionalmente.