\chapter{Analisi dei requisiti}
\label{cap:analisi-requisiti}

\intro{Il capitolo è dedicato all'analisi dei requisiti della piattaforma, con l'obiettivo di fornire una visione completa e dettagliata delle funzionalità e delle caratteristiche attese dal sistema. Verranno illustrate le esigenze degli utenti e del contesto operativo, evidenziando le specifiche tecniche e le funzionalità che hanno guidato le scelte progettuali.}\\

%\section{Casi d'uso}

%Per lo studio dei casi di utilizzo del prodotto sono stati creati dei diagrammi.
%I diagrammi dei casi d'uso (in inglese \emph{Use Case Diagram}) sono diagrammi di tipo \gls{uml}\glsfirstoccur dedicati alla descrizione delle funzioni o servizi offerti da un sistema, così come sono percepiti e utilizzati dagli attori che interagiscono col sistema stesso. \\
%Nel contesto del progetto, volto alla creazione di una piattaforma di monitoraggio per applicazioni \emph{web} con strumenti di \gls{apmg} e componenti di intelligenza artificiale per l’analisi automatica dei dati, le interazioni da parte dell’utente sono ridotte allo stretto necessario. Questo approccio minimizza l’intervento manuale, garantendo che le funzionalità principali siano accessibili in maniera intuitiva e diretta. Di conseguenza, i diagrammi dei casi d’uso risultano semplici e in numero limitato, focalizzati sulle operazioni essenziali per la raccolta, l’analisi e la visualizzazione dei dati di monitoraggio.

%\begin{figure}[!h] 
    %\centering 
    %\includegraphics[width=0.9\columnwidth]{usecase/scenario-principale} 
    %\caption{Use Case - UC0: Scenario principale}
%\end{figure}

%\begin{usecase}{0}{Scenario principale}
%\usecaseactors{Sviluppatore applicativi}
%\usecasepre{Lo sviluppatore è entrato nel plug-in di simulazione all'interno dell'IDE}
%\usecasedesc{La finestra di simulazione mette a disposizione i comandi per configurare, registrare o eseguire un test}
%\usecasepost{Il sistema è pronto per permettere una nuova interazione}
%\label{uc:scenario-principale}
%\end{usecase}

\section{Requisiti del sistema}

A seguito di un'attenta attività di analisi del progetto e degli obiettivi tecnici e funzionali prefissati, sono state redatte le tabelle di tracciamento che riassumono in modo strutturato i requisiti individuati. \\
Durante questa fase sono state identificate differenti tipologie di requisiti, distinte sia in base alla loro categoria (funzionale, non funzionale, qualitativo o di vincolo), sia in base alla loro priorità di implementazione (obbligatorio, desiderabile o opzionale).
Per garantire una tracciabilità chiara e univoca, a ciascun requisito è stato assegnato un codice identificativo composto da lettere che ne descrivono la tipologia e l'importanza, secondo la seguente convenzione:
\begin{description}
	\item [R =] Requisito
	\item [F =] Funzionale
    \item [N =] Non funzionale
    \item [Q =] Qualitativo
    \item [V =] Vincolo
    \item [O =] Obbligatorio
    \item [D =] Desiderabile
    \item [Z =] Opzionale
\end{description}

\newpage

Ogni requisito analizzato sarà identificato univocamente dalla seguente notazione:
\begin{center}
    \textbf{R[Priorità][Categoria]-[Numero]}
\end{center}
Dove:

\begin{itemize}
    \item \textbf{Priorità} indica la priorità di implementazione, che può essere O = Obbligatorio, D = Desiderabile, Z = Opzionale;
    \item \textbf{Categoria} la categoria di appartenenza, che può essere F = Funzionale, N = Non funzionale, Q = Qualitativo, V = Vincolo;
    \item \textbf{Numero} un numero progressivo che identifica in modo univoco il requisito all'interno della sua categoria.
\end{itemize}
Nelle tabelle \ref{tab:requisiti-funzionali}, \ref{tab:requisiti-non-funzionali}, \ref{tab:requisiti-qualitativi} e \ref{tab:requisiti-vincolo} sono riportati in modo sintetico tutti i requisiti emersi dall'analisi, classificati in base alla loro priorità e accompagnati da una breve descrizione della relativa funzionalità o vincolo tecnico. \\


\newpage
\subsection{Requisiti funzionali}
I requisiti funzionali descrivono cosa deve fare il sistema.
Sono le funzionalità concrete che la soluzione deve offrire per raggiungere gli obiettivi del progetto.

\begin{table}[h]
\caption{Tabella del tracciamento dei requisiti funzionali}
\label{tab:requisiti-funzionali}
\begin{tabularx}{\textwidth}{lXl}
\hline
\rowcolor[gray]{0.8}
\textbf{Codice} & \textbf{Descrizione} & \textbf{Classificazione}\\
\hline
ROF-1     & Il sistema deve permettere la raccolta automatica di metriche e \emph{log} relativi alla \emph{web application} tramite agenti \emph{OpenTelemetry} o \emph{Elastic APM}. & Obbligatorio \\
\hline

\hline
ROF-2     & Il sistema deve inviare i dati raccolti agli \emph{endpoint} \emph{Elasticsearch} per l'analisi e l'indicizzazione. & Obbligatorio \\
\hline

\hline
ROF-3     & Il sistema deve consentire la creazione di \emph{pipeline} di \emph{log} tramite \emph{Logstash} per filtraggio, trasformazione e inoltro dei dati in \emph{Elasticsearch}. & Obbligatorio \\
\hline

\hline
ROF-4     & Il sistema deve prevedere la configurazione di \emph{Elastic Agents} (\emph{Beats}/\emph{\gls{apm}}) per la raccolta dati della navigazione. & Obbligatorio \\
\hline

\hline
ROF-5     & Il sistema deve generare \emph{dashboard} avanzate e visualizzazioni in \emph{Kibana}, con metriche di \emph{performance}, accesso e flussi utente. & Obbligatorio \\
\hline

\hline
ROF-6     & Il sistema deve permettere la verifica della corretta acquisizione dei dati e la loro indicizzazione in \emph{Elasticsearch}. & Obbligatorio \\
\hline

\hline
ROF-7     & Il sistema deve prevedere lo sviluppo e l'esecuzione di \emph{script} automatizzati in \emph{Python} o \emph{Java} per la simulazione del traffico utente (\emph{Synthetic Monitoring}). & Obbligatorio \\
\hline

\hline
ROF-8     & Deve essere possibile filtrare e ricercare i \emph{log} per \emph{host}, servizio, livello di severità o periodo temporale. & Obbligatorio \\
\hline

\hline
RDF-9     & Il sistema dovrebbe prevedere la configurazione di regole di \emph{alerting} e notifiche in tempo reale per anomalie rilevate. & Desiderabile \\
\hline

\hline
RDF-10     & Il sistema dovrebbe integrare algoritmi di \emph{Machine Learning} per l'individuazione automatica di anomalie. & Desiderabile \\
\hline

\hline
RDF-11     & Il sistema dovrebbe consentire l'esportazione delle \emph{dashboard} o dei risultati delle \emph{query} in formato \gls{pdf}\glsfirstoccur o \gls{csv}\glsfirstoccur. & Desiderabile \\
\hline

\hline
RZF-12     & Il sistema può prevedere un modulo aggiuntivo per la generazione automatica di report periodici delle metriche raccolte. & Opzionale \\
\hline

\hline
RZF-13     & Il sistema può consentire l'importazione automatica delle configurazioni \gls{apm} da ambienti di \emph{test} o \emph{staging}. & Opzionale \\
\hline

\end{tabularx}
\end{table}%

\newpage
\subsection{Requisiti non funzionali}
I requisiti non funzionali definiscono come il sistema deve comportarsi, cioè le sue proprietà di qualità interna.
Non aggiungono nuove funzioni, ma impongono vincoli di prestazioni, sicurezza, disponibilità, scalabilità, affidabilità e manutenibilità.

\begin{table}[h]
\caption{Tabella del tracciamento dei requisiti non funzionali}
\label{tab:requisiti-non-funzionali}
\begin{tabularx}{\textwidth}{lXl}
\hline
\rowcolor[gray]{0.8}
\textbf{Codice} & \textbf{Descrizione} & \textbf{Classificazione}\\
\hline
RON-1    & Il sistema deve essere scalabile e consentire l'aggiunta di nuove fonti di dati o agenti senza compromettere la stabilità. & - Obbligatorio \\
\hline

\hline
RON-2    & Il sistema deve garantire l'affidabilità nella trasmissione e nella conservazione dei dati raccolti. & - Obbligatorio \\
\hline

\hline
RON-3    & La piattaforma deve assicurare un tempo di latenza accettabile nella visualizzazione dei dati (< 5 secondi per l'aggiornamento delle \emph{dashboard}). & - Obbligatorio \\
\hline

\hline
RDN-4    & Il sistema dovrebbe garantire la possibilità di eseguire \emph{test} di carico e stress per valutare la stabilità dell'ambiente. & - Desiderabile \\
\hline

\hline
RDN-5    & Il sistema dovrebbe supportare l'autenticazione per la gestione degli accessi a \emph{Kibana}. & - Desiderabile \\
\hline


\end{tabularx}
\end{table}%


\newpage
\subsection{Requisiti qualitativi}
I requisiti qualitativi specificano le proprietà qualitative che influenzano l'esperienza d'uso e la manutenibilità.
Si concentrano su aspetti percepibili, come semplicità, chiarezza, flessibilità o estendibilità.

\begin{table}[h]
\caption{Tabella del tracciamento dei requisiti qualitativi}
\label{tab:requisiti-qualitativi}
\begin{tabularx}{\textwidth}{lXl}
\hline
\rowcolor[gray]{0.8}
\textbf{Codice} & \textbf{Descrizione} & \textbf{Use Case}\\
\hline
ROQ-1    & L'interfaccia di \emph{Kibana} deve offrire una rappresentazione chiara e intuitiva delle metriche principali. & - Obbligatorio \\
\hline

\hline
ROQ-2    & I dati devono essere visualizzabili in forma aggregata e filtrabile in base a intervalli temporali e categorie di evento. & - Obbligatorio\\
\hline

\hline
ROQ-3    & Le \emph{dashboard} devono presentare una chiara distinzione cromatica tra metriche positive, neutre e anomale. & - Obbligatorio\\
\hline

\hline
RDQ-4    & Le \emph{dashboard} dovrebbero essere personalizzabili dall'utente secondo criteri di interesse (\emph{performance}, accessi, flussi). & - Desiderabile \\
\hline

\hline
RZQ-5    & Il sistema può includere un \emph{layout dark/light mode} o temi grafici personalizzati per una migliore leggibilità. & - Opzionale \\
\hline

\end{tabularx}
\end{table}%

\newpage
\subsection{Requisiti di vincolo}
Impongono limitazioni o condizioni esterne al progetto: ambienti, tecnologie, compatibilità, strumenti, standard aziendali o legali.

\begin{table}[H]
\caption{Tabella del tracciamento dei requisiti di vincolo}
\label{tab:requisiti-vincolo}
\begin{tabularx}{\textwidth}{lXl}
\hline
\rowcolor[gray]{0.8}
\textbf{Codice} & \textbf{Descrizione} & \textbf{Use Case}\\
\hline
ROV-1    & Il sistema deve utilizzare i prodotti della suite \emph{Elastic Stack} (\emph{Elasticsearch, Logstash, Kibana, Beats, APM Server/Agent}). & - Obbligatorio \\
\hline

\hline
ROV-2    & L'ambiente operativo deve essere \emph{Linux} (\emph{Red Hat} o distribuzioni equivalenti). & - Obbligatorio \\
\hline

\hline
ROV-3    & Tutti i componenti \emph{software} devono essere compatibili con la versione di \emph{Linux} installata sull'ambiente aziendale (es. \emph{Ubuntu 22.04 LTS} o \emph{Red Hat 9}). & - Obbligatorio \\
\hline

\hline
ROV-4    & Le componenti devono rispettare le seguenti versioni minime:

\begin{itemize}
    \item \emph{Python} >= 3.10
    \item \emph{Java} >= 16
    \item \emph{Node.js} >= 17
    \item \emph{Logstash} >= 8.10
    \item \emph{Kibana} >= 8.10
    \item \emph{Beats} >= 8.10
    \item \emph{APM Server} >= 8.10
    \item \emph{Elasticsearch} >= 8.10
    \item \emph{OpenTelemetry} >= 1.0
\end{itemize}& - Obbligatorio \\
\hline

\hline
ROV-5    & La \emph{web application} deve essere compatibile con i principali \emph{browser} (\emph{Chrome} >= 120, \emph{Firefox} >= 115, \emph{Edge} >= 120, \emph{Apple Safari} >= 15). & - Obbligatorio \\
\hline

\hline
ROV-6    & Tutte le configurazioni devono essere eseguite in un ambiente \emph{Docker} o su infrastruttura fornita da Kirey Group. & - Obbligatorio \\
\hline

\hline
RDV-7    & La documentazione tecnica deve essere redatta in formato \emph{Markdown}, \emph{LaTeX} o \gls{pdf}, seguendo gli standard aziendali. & - Desiderabile \\
\hline

\hline
RDV-8    & Il sistema dovrebbe consentire la distribuzione automatizzata tramite \emph{Docker Compose}. & - Desiderabile \\
\hline

\end{tabularx}
\end{table}

\newpage
\section{Riepilogo dei requisiti}

\begin{center}
\captionof{table}{Riepilogo dei requisiti}
\label{tab:requisiti-riepilogo}
\begin{tabular}{|l|c|c|c|c|}
\hline
\rowcolor[gray]{0.8}
\textbf{Tipologia} & \textbf{Obbligatorio} & \textbf{Desiderabile} & \textbf{Opzionale} & \textbf{Totale} \\
\hline
\textbf{Funzionali} & 8 & 3 & 2 & 13 \\
\hline

\hline
\textbf{Non funzionali} & 3 & 2 & 0 & 5 \\
\hline

\hline
\textbf{Qualitativi} & 3 & 1 & 1 & 5 \\
\hline

\hline
\textbf{Di Vincolo} & 6 & 2 & 0 & 8 \\
\hline

\hline
\textbf{Totale} &  &  &  & 31 \\
\hline

\end{tabular}
\end{center}

\newpage
\section{Rappresentazione architetturale preliminare}
La rappresentazione architetturale descrive la struttura della soluzione proposta, evidenziando le principali componenti coinvolte nel sistema di \gls{apm} e le loro relazioni. \\
Essa costituisce il collegamento tra la fase di analisi dei requisiti e la successiva implementazione pratica. I dettagli tecnici e i diagrammi della soluzione verranno approfonditi nel capitolo dedicato alla progettazione architetturale.


\subsection{Struttura generale della soluzione di monitoraggio}
La soluzione di \gls{apm} proposta si basa su un'architettura modulare che integra diversi componenti dell'\emph{Elastic Stack} e strumenti di osservabilità \emph{open source}.
Gli agenti \textit{OpenTelemetry} installati sull'applicazione \textit{Spring PetClinic} raccolgono metriche e tracce relative all'esecuzione delle richieste, che vengono inviate all'\textit{APM Server}.
Quest'ultimo elabora e normalizza i dati, inoltrandoli a \textit{Elasticsearch} per l'indicizzazione e la persistenza.
I dati così organizzati vengono infine consultati e analizzati tramite \textit{Kibana}, che fornisce una visualizzazione interattiva delle metriche e la possibilità di configurare \emph{dashboard} e regole di \emph{alerting}.


\subsection{Integrazione dei componenti principali}
L'architettura della soluzione di monitoraggio si compone dei seguenti moduli principali:
\begin{itemize}
    \item \textbf{OpenTelemetry:} fornisce gli agenti per la raccolta di metriche, tracce e \emph{log} dall'applicazione \emph{Spring PetClinic}. Questi agenti sono integrati direttamente nel codice dell'applicazione e inviano i dati all'\emph{APM Server};
    \item \textbf{Elastic APM Server:} riceve i dati dalle sonde \emph{OpenTelemetry} e li converte in un formato compatibile con \emph{Elasticsearch};
    \item \textbf{Logstash:} funge da \emph{pipeline} di pre-elaborazione: filtra, arricchisce e reindirizza i dati, ad esempio separando \emph{log} applicativi, metriche di \emph{performance} e tracce di richiesta;
    \item \textbf{Elasticsearch:} immagazzina i dati indicizzati e li rende interrogabili in tempo reale;
    \item \textbf{Kibana:} consente la visualizzazione e l'analisi delle metriche, oltre alla configurazione di \emph{dashboard}, avvisi e \emph{query} analitiche.
\end{itemize}

(Immagine interazione APM)


\subsection{Applicazione di riferimento: Spring PetClinic}
Per la sperimentazione del sistema di \gls{apm} è stata utilizzata come applicazione di riferimento \emph{Spring PetClinic}, una \emph{web application} \emph{open source} sviluppata in \emph{Java} e basata sul \emph{framework} \emph{Spring Boot}.  
La scelta di \emph{PetClinic} è motivata dalla sua architettura chiara e ben documentata, composta da un livello di presentazione (\emph{controller} e \emph{template}), un livello logico (servizi) e un livello di persistenza (\emph{repository} e \emph{database} \gls{mysqlg}\glsfirstoccur).  
Tale struttura consente di osservare comportamenti applicativi realistici, come chiamate \gls{httpg}\glsfirstoccur, interazioni con il \emph{database} e logiche di \emph{business}, fornendo un contesto ideale per testare le funzionalità di monitoraggio offerte da \emph{Elastic APM} e \emph{OpenTelemetry}.  

\begin{figure}[!h] 
    \centering 
    \includegraphics[width=\columnwidth]{Architettura_PetClinic.png} 
    \caption{Figura 3.1 - Architettura di Spring PetClinic}
\end{figure}


L'applicazione \emph{PetClinic} verrà quindi utilizzata come caso di studio per la validazione della soluzione di monitoraggio proposta, analizzando successivamente il modo in cui i diversi moduli architetturali interagiscono con essa.



\subsection{Metriche osservate e casi d'uso}
I casi d'uso descrivono le principali interazioni tra l'utente del sistema di \gls{apm} e la piattaforma di monitoraggio. \\
Nel contesto del progetto, l'attore principale è l'\emph{Observability Engineer}, ossia la figura responsabile dell'analisi delle prestazioni applicative, della configurazione delle \emph{dashboard} e dell'interpretazione delle metriche raccolte tramite \emph{Elastic APM}, \emph{Logstash} ed \emph{Elasticsearch}.  
Tale figura si occupa di garantire la corretta osservabilità dell'applicazione monitorata, identificando tempestivamente anomalie e ottimizzando il comportamento del sistema. \\
La soluzione di \gls{apm} sviluppata è destinata a un'utenza tecnica interna, composta da personale specializzato nell'ambito dell'\emph{observability}. \\
Le interazioni descritte di seguito rappresentano scenari tipici di utilizzo delle funzionalità fornite da \emph{Kibana} e dagli strumenti di monitoraggio associati, con particolare riferimento all'applicazione \emph{Spring PetClinic} utilizzata come caso di studio.

\vspace{1em}

\begin{figure}[!h] 
    \centering 
    \includegraphics[width=0.9\columnwidth]{usecase/scenario-principale} 
    \caption{Use Case - UC0: Scenario principale}
\end{figure}

\begin{usecase}{0}{Analisi delle prestazioni applicative}
\usecaseactors{Observability Engineer}
\usecasepre{Il sistema \gls{apm} è correttamente configurato e sta ricevendo metriche dall'applicazione \emph{Spring PetClinic}}
\usecasedesc{L'utente accede a \emph{Kibana} e consulta la dashboard dedicata alle metriche di \emph{performance} per analizzare i tempi medi di risposta degli \emph{endpoint}, individuando eventuali colli di bottiglia o aumenti anomali della latenza}
\usecasepost{Le metriche sono interpretate e vengono individuate le aree critiche su cui intervenire}
\label{uc:analisi-prestazioni}
\end{usecase}

\vspace{0.5em}

\begin{usecase}{1}{Monitoraggio delle risorse del sistema}
\usecaseactors{Observability Engineer}
\usecasepre{Le sonde di \emph{OpenTelemetry} e gli agenti \emph{Elastic APM} sono attivi sull'ambiente di esecuzione}
\usecasedesc{L'utente consulta la \emph{dashboard} di sistema per monitorare l'utilizzo di \gls{cpug}\glsfirstoccur, memoria e connessioni attive, verificando la corretta gestione delle risorse e la stabilità dell'ambiente}
\usecasepost{L'utente ottiene una panoramica aggiornata dello stato del sistema, utile per valutare l'efficienza dell'infrastruttura}
\label{uc:monitoraggio-risorse}
\end{usecase}

\vspace{0.5em}

\begin{usecase}{2}{Rilevazione di errori applicativi}
\usecaseactors{Observability Engineer}
\usecasepre{Le metriche di log e tracing sono inviate correttamente a \emph{Elasticsearch} tramite \emph{Logstash}}
\usecasedesc{L'utente analizza in \emph{Kibana} i log applicativi e le tracce distribuite per identificare errori o eccezioni ricorrenti, filtrandoli per servizio o \emph{endpoint}}
\usecasepost{Gli errori vengono tracciati e catalogati, permettendo di individuare le cause e pianificare interventi correttivi}
\label{uc:rilevazione-errori}
\end{usecase}

\vspace{0.5em}

\begin{usecase}{3}{Gestione delle regole di alerting}
\usecaseactors{Observability Engineer}
\usecasepre{Le dashboard e le metriche principali sono già configurate in \emph{Kibana}}
\usecasedesc{L'utente definisce soglie di allarme per tempi di risposta, errori \gls{httpg} o utilizzo eccessivo delle risorse. In caso di superamento dei valori critici, \emph{Kibana} genera notifiche o invia alert via e-mail}
\usecasepost{Il sistema invia automaticamente notifiche di anomalia, consentendo un intervento tempestivo per ripristinare le condizioni ottimali}
\label{uc:gestione-alerting}
\end{usecase}

\vspace{1em}

Queste informazioni saranno rappresentate tramite \emph{dashboard} dedicate in Kibana, che consentono una visualizzazione interattiva e la definizione di soglie di \emph{alerting} personalizzate.


%\section{Conclusioni}
%Il presente capitolo ha illustrato l'analisi dei requisiti del sistema di \gls{apm}, la loro classificazione e la relativa mappatura sulle componenti principali dell'architettura prevista.  
