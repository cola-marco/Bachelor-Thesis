\cleardoublepage
\phantomsection
\pdfbookmark{Sommario}{Sommario}
\begingroup
\let\clearpage\relax
\let\cleardoublepage\relax
\let\cleardoublepage\relax

\chapter*{Sommario}

Il presente documento descrive il lavoro svolto durante il periodo di stage, della durata di 320 ore, dal laureando Marco Cola presso l'azienda Kirey S.r.l. \\
L'obiettivo principale dello stage è stato lo sviluppo di una piattaforma per il monitoraggio e l'analisi dei dati di navigazione relativi ad una \emph{web application}, con particolare attenzione alla raccolta, indicizzazione ed elaborazione dei \emph{log}. \\
Il progetto si è articolato in quattro fasi principali:

\begin{itemize}
    \item Una prima fase di preparazione dell'ambiente di lavoro, comprendente l'individuazione, l'installazione e la configurazione delle componenti \emph{software} necessarie, con successiva verifica del corretto funzionamento e della connettività tra i moduli;
    \item Una seconda fase di implementazione dell'estrazione dei dati, realizzata attraverso la configurazione di agenti di raccolta, la creazione di \emph{pipeline} di \emph{log} tramite \emph{Logstash} e l'invio dei dati a \emph{Elasticsearch}, con validazione del processo di acquisizione e indicizzazione;
    \item Una terza fase di elaborazione e rappresentazione grafica dei dati, che ha previsto lo sviluppo di \emph{script} per la generazione di traffico utente e monitoraggio, l'analisi e aggregazione dei dati su \emph{Elasticsearch} e la realizzazione di dashboard avanzate in \emph{Kibana} con metriche di performance, accesso e flussi utente. Sono state inoltre configurate regole di \emph{alerting} e notifiche per la rilevazione in tempo reale di anomalie mediante \emph{Machine Learning};
    \item Una fase finale di documentazione tecnica del progetto, contenente la descrizione delle tecnologie e dei prodotti utilizzati, la rappresentazione dei flussi logici dell'applicazione, nonché un'analisi dei pro e contro di ciascuna componente e delle principali criticità riscontrate.
\end{itemize}
Lo stage ha permesso di acquisire competenze trasversali nell'ambito dell'osservabilità delle applicazioni \emph{web}, approfondendo l'utilizzo dello \emph{stack} \emph{Elastic} (\emph{Elasticsearch, Logstash, Kibana}) e l'integrazione con strumenti di monitoraggio automatizzato, con un approccio volto alla creazione di soluzioni scalabili, robuste e orientate al miglioramento continuo delle prestazioni.
%\vfill

%\selectlanguage{english}
%\pdfbookmark{Abstract}{Abstract}
%\chapter*{Abstract}

%\selectlanguage{italian}

\endgroup

\vfill
